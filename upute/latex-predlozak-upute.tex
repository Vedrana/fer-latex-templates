% Upute za LaTeX predložak ZR-a i Seminara, verzija: 0.5beta

\documentclass{zavrsni}
\usepackage{algorithm}
\usepackage{algorithmic}
\usepackage{listings}
\usepackage{longtable}
\usepackage{fancybox}
\lstset{language=Java, tabsize=2}

\usepackage{tikz}
\usetikzlibrary{shapes,shadows}
\tikzstyle{abstractbox} = [draw=black, fill=white, rectangle,
inner sep=10pt, style=rounded corners, drop shadow={fill=black,
opacity=1}]
\tikzstyle{abstracttitle} =[fill=white]

\newcommand{\boxresult}[2][fill=white]{
  \begin{center}
    \begin{tikzpicture}
      \node [abstractbox, #1] (box){%
      	\begin{minipage}{0.90\linewidth}
          \footnotesize #2
        \end{minipage}
	  };
      \node[abstracttitle, right=10pt] at (box.north west) {Rezultat};
    \end{tikzpicture}
  \end{center}
}

\begin{document}
\thesisnumber{000}
\title{Upute za korištenje \LaTeX{} predloška za Završni i Diplomski rad te Seminar}
\author{Ivan Krišto \and Boran Car \and Mateja Čuljak \and Vedrana Janković}

\maketitle

\tableofcontents

\chapter{Uvod}
Upute prvenstveno služe približavanju \LaTeX-a što široj populaciji studenata.
Namjena uputa je ukratko objasniti način korištenja predložaka za pismene radove
te dati kratke upute i savjete oko izrade dokumenata u \LaTeX-u. Upute su vođene
problemima vezanima uz formatiranje teksta Završnog i Diplomskog rada te Seminara
koji su obrađeni jednostavnim primjerima.

\LaTeX{} je sustav za visokokvalitetno slovoslagarstvo. Omogućava jednostavnu i
kvalitetnu izradu raznovrsnih tekstova. \LaTeX{} je zapravo dodatak \TeX-a, sustava
kojeg je još 1976.\ godine razvio izuzetno utjecajan matematičar i računarski
znanstvenik, Donald E.~Knuth. Sam \LaTeX{} je također napisala utjecajna
znanstvena ličnost, Leslie Lamport.

Osnovna ideja \LaTeX-a je koncentriranje posla izrade dokumenata na strukturu i
sadržaj, a ne na izgled.\footnote{Sličan princip spada u osnove programskog inženjerstva.}
Primjerice, izgled dokumenta napisanog sadržaja i definirane strukture, 
mijenja se jednostavnom izmjenom predloška.\footnote{Predložak je skup
definicija stila.}

Bitno je napomenuti da \LaTeX{} omogućava jednostavnu izradu lijepih dokumenata
po predefiniranim stilovima. Veći stilski zahvati znaju biti izuzetno teški (čak i
izrada ovog predloška) i zahtijevaju bolje poznavanje \LaTeX-a ili čak samog
\TeX-a. Navedena činjenica se ne smije gledati isključivo u negativnom kontekstu jer
ta nuspojava dizajna \TeX{} sustava često osigurava poštivanje tipografskih principa
koji su ugrađeni u same temelje \TeX-a i \LaTeX-a. Složenost primjene većih stilskih
zahvata je usko vezana s mogućnostima koje \TeX{} nudi, a one daleko nadilaze sve
trenutno dostupne tzv.\ \emph{WYSIWYG}\footnote{What You See Is What You Get} tekst
procesore poput \emph{Microsoft Worda}, \emph{Word Perfecta} ili \emph{OpenOffice Writera}.

Osobe neupućene u rad s \LaTeX-om možda najviše zanima usporedba \LaTeX-a i alata
poput \emph{Microsoft Worda} i \emph{OpenOffice Writera}. U svrhu poticanja
korištenja \LaTeX-a navedene su prednosti \LaTeX-a nad danas uobičajenim tekst
procesorima. Osnovna prednost je već spomenuto odjeljivanje sadržaja i strukture
od stila. Navedeno omogućava konzistentnost stila kroz dokument i olakšava
publikaciju dokumenata. Sljedeća bitna prednost je činjenica da se \TeX{}
dokumenti pišu kao obične tekstualne datoteke sa \TeX{} sintaksom. To omogućava
prenosivost dokumenta kroz verzije \TeX-a (primjerice, dokument napisan MS Wordom
2007 nije čitljiv u MS Wordu 2000). Navedena činjenica je tim bitnija što
olakšava pisanje velikih dokumenata, primjerice, korištenjem \LaTeX-ove
mogućnosti da se dokument sastoji od više datoteka (radi se o korištenju
\verb|\include{}| naredbe, vidi:
\url{http://en.wikibooks.org/wiki/LaTeX/Multiple_files})
ili dokumenata koje istovremeno piše više ljudi (knjige, razne korisničke upute i
sl.) korištenjem sustava za verzioniranje (VCS sustavi poput SVN-a, Mercuriala i
Gita).\footnote{Ove upute su razvijene korištenjem SVN VCS sustava te su dostupne
na \url{http://code.google.com/p/fer-latex-templates/}} Više o sustavima za
verzioniranje i \LaTeX-u pročitajte na
\url{http://en.wikibooks.org/wiki/LaTeX/Collaborative_Writing_of_LaTeX_Documents}.

Za dodatne argumente u korist \LaTeX-a konzultirajte \citep{taraborelli2009beauty}
i \citep{cottrell1999word}.

Cilj je ovih uputa omogućiti korisniku što bezbolniju izradu Diplomskog rada, Završnog
rada ili Seminara korištenjem \LaTeX-a. Dane su osnovne potrebne informacije te poveznice
za daljnje upoznavanje s \LaTeX-om. Kao osnovnu literaturu za upoznavanje s \LaTeX-om
iskoristite \url{http://en.wikibooks.org/wiki/LaTeX/} te knjige \citep{oetiket2007lshort}
i \citep{ungar2002uvod}. Mislilo se i na iskusnije korisnike \LaTeX-a te su upute koncipirane
na način da napredni korisnici trebaju pročitati samo \ref{ch:upute} i
\ref{ch:stil-i-struktura} poglavlje te odjeljke koji su eksplicitno navedeni u ta dva poglavlja.
Novim korisnicima \LaTeX-a se predlaže da preskoče \ref{ch:upute} poglavlje te se koncentriraju
na \ref{ch:latex-upute} poglavlje, odnosno odjeljke koji ih zanimaju (primjerice, ako
u svojem radu nemate potrebu za korištenjem tablica ili dodatnih paketa, nemojte se zamarati
čitanjem tih odjeljaka).


\chapter{Upute za korištenje}
\label{ch:upute}
% Kratko i jasno za ljude koji znaju koristiti LaTeX.
Osnova predloška je \texttt{cls} datoteka. Uz navedenu datoteku potrebna
je \texttt{tex} datoteka za sadržaj rada te \texttt{bib} datoteka za literaturu. Sve
navedene datoteke moraju se nalaziti u istom direktoriju.

Razvijena su tri predloška, predlošci za Završni i Diplomski rad te za Seminar.
Radi jednostavnosti, konkretnosti i sličnosti među razvijenim predlošcima 
upute će se uvijek referencirati na Završni rad i datoteku \texttt{zavrsni.cls}, 
iako sve navedeno vrijedi i za ostale predloške. Razlike u predlošcima su
navedene u ovom poglavlju ispod primjera uporabe.

Povezivanje \texttt{tex} dokumenta i predloška iz \texttt{zavrsni.cls} datoteke
vrši se naredbom \verb|\documentclass{zavrsni}| koja mora biti prva naredba u
dokumentu. Za Diplomski rad koristite datoteku \texttt{diplomski.cls} i razred
\texttt{diplomski}, a za Seminar \texttt{ferseminar.cls} i razred
\texttt{ferseminar}.\footnote{Razred ``seminar'' već postoji unutar \LaTeX-a i
namjenjen je izradi prezentacija.}

Predložak, osim standardnih polja kao što \texttt{author} i \texttt{title}
sadrži obavezno polje \texttt{thesisnumber} -- broj Završnog rada. Osnovna
struktura \texttt{tex} datoteke mora biti:
\begin{verbatim}
\documentclass{zavrsni}

% Ovdje možete staviti dodatne pakete koji
% su vam potrebni, npr.:
\usepackage{longtable}

\begin{document}
\thesisnumber{000}
\title{Naslov rada}
\author{Vaše ime i prezime}
\maketitle
\tableofcontents
% Tu možete staviti popis slika i tablica

\chapter{Uvod}
% Sadržaj rada, ostala poglavlja i odjeljci.

\bibliography{literatura}
\bibliographystyle{plainnat}

% Dodatak nije obavezan
\appendix
\chapter{Poglavlje pod dodatkom}

\begin{sazetak}
Sažetak rada.

\kljucnerijeci{<popis ključnih riječi>}
\end{sazetak}

\engtitle{Naslov rada na engleskom jeziku}
\begin{abstract}
Sažetak na engleskom jeziku.

\keywords{<popis ključnih riječi na engleskom>}
\end{abstract}

\end{document}
\end{verbatim}

Ukoliko koristite predložak za Diplomski rad ili Seminar, umjesto naredbe \verb|\documentclass{zavrsni}|
morate navesti \verb|\documentclass{diplomski}|, odnosno \verb|\documentclass{ferseminar}|.
Uz to, Seminar nema ``broj rada,'' tj.\ ne postoji naredba \verb|\thesisnumber{}|, ali je potrebno navesti
voditelja rada naredbom \verb|\voditelj{}| (tj.\ umjesto naredbe \verb|\thesisnumber{}| stavite naredbu
\verb|\voditelj{}| i kao argument navedite ime i prezime vašeg mentora ili mentorice, odnosno voditelja
vašeg rada).

Predložak se temelji na \emph{report} razredu te su gradivni elementi sadržaja
dokumenta (hijerarhijski navedeno): \emph{chapter}, \emph{section}, \emph{subsection} i
\emph{subsubsection}.

O stilu i strukturi dokumenta pogledajte poglavlje \ref{ch:stil-i-struktura}, a
za upoznavanje s \LaTeX-om te nekim njegovim posebnostima i receptima pogledajte
poglavlje \ref{ch:latex-upute} Ako želite koristiti drugačiji font od
trenutnog, u datoteci \texttt{zavrsni.cls} promijenite liniju
\verb|\RequirePackage{lmodern}| u npr., \verb|\RequirePackage{times}| za font
Times New Roman.

Napravljeni predložak sadrži rješenje za dodavanje sažetka rada i posvete te
naredbu \texttt{engl}. Za navedeno pogledajte odjeljke \ref{sec:sazetak},
\ref{sec:posveta} i \ref{sec:engl} Diplomski i Završni rad moraju imati naveden
sažetak, dok Seminar ne mora.

Za pakete uključene u sam predložak pogledajte polja \texttt{RequirePackage} u
datoteci \texttt{završni.cls}.

\chapter{Stil i struktura}
\label{ch:stil-i-struktura}
Stil Završnog rada (format, font, prored, margine i sl.) određen je predloškom te
se nije potrebno brinuti o njemu. U nastavku će detaljnije biti opisana struktura
Završnog rada preuzeta iz uputa na stranici
\url{http://www.fer.hr/predmet/zavrad}. Sve navedeno za Završni rad vrijedi i za Diplomski.

Na početak završnog rada uvezuje se \emph{naslovna stranica}, \emph{izvornik
Završnog zadatka} (dokument koji podižete na vašem Zavodu), \emph{stranica sa zahvalom}
(ili prazna stranica) te ostatak rada (naslovna stranica se broji kao prva te
je stranica sa popisom sadržaja četvrta, stoga je označena rimskim brojem ``iv'').

\section{Osnovna poglavlja Završnog rada}
Osnovna poglavlja su:
\begin{enumerate}
\item uvod,
\item poglavlja vezana uz temu (naslovi poglavlja dogovaraju se s mentorom),
\item zaključak (kandidat vlastoručno potpisuje Završni rad iza zaključka),
\item literatura,
\item naslov, sažetak i ključne riječi (na hrvatskom i engleskom).
\end{enumerate}
Završni rad mora imati sažetak od stotinjak riječi. Prilikom navođenja
ključnih riječi izbjegavajte pojmove koji su već navedeni u naslovu rada.

\section{Ostale upute}
U svim dokumentima obavezno primjenjivati \emph{SI} jedinice. Slike, formule i tablice
treba numerirati. Opis tablice stavlja se \emph{iznad tablice}, a opis slike
\emph{ispod slike}. U opisu slike ili tablice pišu se samo podaci neophodni za njeno
razumijevanje. Dodatna objašnjenja daju se u tekstu uz povezivanje sa slikom ili
tablicom. Osi i parametri na slikama i grafičkim prikazima trebaju biti
obilježeni. Daljnji opis tog grafičkog prikaza treba se nalaziti u tekstu rada.
Formule se obilježavaju brojevima u zagradi, uz desni rub stranice, a u tekstu se
poziva na broj formule.

\section{Privitak završnog rada}
Privitak može činiti tehnička dokumentacija vezana uz završni rad (npr.\
električka i položajna shema sklopa, sastavnica, predložak tiskane veze i sl.).

\chapter{Upute za \LaTeX}
\label{ch:latex-upute}
U nastavku su dane upute za \LaTeX, no sužene za potrebe brzog uvoda u rad s
\LaTeX-om i konkretnim predloškom.

Ako se želite bolje upoznati s \LaTeX-om, konzultirajte poveznice navedene u
dodatku \ref{ch:korisne-stranice} te knjige poput \citep{oetiket2007lshort}.

\section{Instalacija \LaTeX{} prevoditelja}
\LaTeX{} dokumenti (dokumenti s \texttt{tex} ekstenzijom) se prevode krajnji
format (npr.\ \texttt{pdf}), tj.\ jezgra produkcije dokumenata je \TeX{}
prevoditelj.\footnote{\LaTeX{} je zapravo samo jedan (malo veći) dodatak
\TeX-a.}

Postoji više prevoditelja (tj.\ distribucija), no za Microsoft Windowse se
preporučuje Mik\TeX{} distribucija dostupna na \url{http://www.miktex.org/}, a za
Linux \TeX{} Live dostupan na \url{http://www.tug.org/texlive/} (koji je dostupan
i za Windowse). Većina Linux distribucija dolazi pripremljena za rad sa
\LaTeX-om, ili je prevoditelja izuzetno jednostavno instalirati preko sistemskog
upravitelja paketima (apt, yum, pacman, \ldots). Za Mik\TeX{} je bitno napomenuti
da samo najnovija verzija može skidati nove pakete, tako da ako se nađete u
situaciji da vam Mik\TeX{} odbija skinuti novi paket koji želite koristiti,
najvjerojatnije je problem u verziji.


\subsection{Prevođenje \LaTeX{} dokumenta}
\LaTeX{} dokumenti su obične tekstualne datoteke s ekstenzijom \texttt{tex}
pisane \TeX{} sintaksom.

Krajnji dokumenti se dobivaju prevođenjem \texttt{tex} datoteka u npr.,
\texttt{pdf} format. Primjerice:
\begin{verbatim}
pdflatex dokument.tex
\end{verbatim}

Svaka \LaTeX{} distribucija sadrži prevoditelje za različite alate
(\texttt{latex} za \texttt{dvi}, \texttt{pdflatex} za \texttt{pdf} te
\texttt{pslatex} za \texttt{ps}) i alate za konverziju između formata (npr.\
\texttt{ps2pdf}).

Prevođenje \LaTeX{} dokumenata je dobro prepustiti razvojnom okruženju radi
razumljivijeg formatiranja ispisa grešaka i upozorenja.

Bitno je napomenuti da je ponekad više puta potrebno pokrenuti prevoditelja da
bi se dobili ispravni rezultati (npr., može se dogoditi da reference nisu
prepoznate, tablica sadržaja nije ispravna i sl.). Razlog tomu je stvaranje
pomoćnih datoteka (aux, bbl, toc i ostale datoteke) koje se
koriste prilikom prevođenja. Prilikom prvog prevođenja, prevoditelj traži te
datoteke te ih, budući da ne postoje, stvara. Prilikom sljedećeg prevođenja,
prevoditelj nalazi potrebne datoteke te ih koristi za generiranje izlaznog
dokumenta.

\section{Razvojna okruženja}
Za pisanje \LaTeX{} dokumenata dovoljan je bilo koji tekst editor, 
primjerice Notepad++ (\url{http://notepad-plus.sourceforge.net/uk/site.htm}) za
Windowse ili Vim (\url{http://www.vim.org/}) za Linux.

Postoji veliki broj \LaTeX{} razvojnih okruženja koja pokušavaju ubrzati rad
korisnika. Tu je bitno napomenuti da većina \LaTeX{} razvojnih okruženja na
Windowsima ne podržava UTF-8 kodnu stranicu!

Među razvojnim okruženjima mogu se istaknuti \emph{Texmaker} i \emph{Eclipse} s
\emph{Texlipse} dodatkom. Navedena razvojna okruženja su besplatna, dostupna
za sve poznatije operacijske sustave te podržavaju UTF-8 kodnu stranicu.
\begin{itemize}
  \item Texmaker -- \url{http://www.xm1math.net/texmaker/}
  \item Eclipse -- \url{http://www.eclipse.org/downloads/} (bilo koji paket);
  Texlipse -- \url{http://texlipse.sourceforge.net/manual/installation.html}.
\end{itemize}

Potrebno je podesiti razvojna okruženja na ispravne putanje od alata iz \TeX{}
distribucije te podesiti kodnu stranicu dokumenata na UTF-8.

Za Texmaker potrebno je ići \emph{Options} $\Rightarrow$ \emph{Configure
Texmaker} $\Rightarrow$ \emph{Editor} $\Rightarrow$ \emph{Editor Font Encoding}
postaviti na UTF-8. Putanje do alata \TeX{} distribucije postavljaju se u
odjeljku \emph{Commands}.

Texlipse postavke podešavaju se unutar Eclipsea. Potrebno je ići na \emph{Window}
$\Rightarrow$ \emph{Preferences} $\Rightarrow$ \emph{Texlipse} $\Rightarrow$
\emph{Builder Settings} te tu podesiti putanje od \TeX{} alata. Kodna stranica
se može podesiti za svaki projekt ili se može iskoristiti uobičajena kodna
stranica. Preporuka je podesiti uobičajenu kodnu stranicu na UTF-8. Za
podešavanje uobičajene kodne stranice potrebno je ići \emph{Window} $\Rightarrow$
\emph{Preferences} $\Rightarrow$ \emph{General} $\Rightarrow$ \emph{Workspace}
$\Rightarrow$ \emph{Text file encoding} postaviti na UTF-8. Za podešavanje
kodne stranice pojedinog projekta, potrebno je otvoriti postavke projekta
\engl{Preferences} te ići \emph{Resource} $\Rightarrow$ \emph{Text file encoding}.

\TeX{} distribucije nude više prevoditelja, \emph{latex}, \emph{pslatex} i
\emph{pdflatex}. Svaki prevoditelj prevodi u svoj format te je preporuka
koristiti direktno \emph{pdflatex} (inače je potrebno koristiti alate kao što
su \emph{ps2pdf} ili \emph{dvipdf} za dobivanje pdf dokumenta kao izlaza).

\section{Osnovno \LaTeX{} formatiranje}
\LaTeX{} dokument se sastoji od dijela sa sadržajem i dijela sa stilom, što čini
upravljanje stilom relativno jednostavnim.

\subsection{Struktura \LaTeX{} dokumenta}
Osnovni dijelovi \LaTeX{} dokumenta su definicija stila naredbom
\verb|\documentclass{}| te okolina \texttt{document}. Naredba
\texttt{documentclass} kao parametar prima definirani stil (predložak za Završni
rad je stil definiran u datoteci \texttt{zavrsni.cls}; ostali često korišteni
stilovi su \emph{article} te \emph{report} na kojem se temelji stil predloška za
Završni rad).

U okolinu \texttt{document} dolazi sadržaj dokumenta. Primjer
osnovne strukture dokumenta možete vidjeti u \ref{ch:upute} poglavlju.

\subsection{Formatiranje sadržaja}
Sadržaj se dijeli na poglavlja \engl{chapter}, odjeljke \engl{section},
pododjeljke \engl{subsection}, podpododjeljke \engl{subsubsection} te
paragrafe. Primjer podjele sadržaja:
\begin{verbatim}
\chapter{Poglavlje 1}
Tekst 1. poglavlja.

\section{Odjeljak 1.1}
Tekst odjeljka 1.1.

\subsection{Pododjeljak 1.1.1}
Tekst pododjeljka.

\subsubsection{Podpododjeljak 1.1.1.1}
Tekst podpododjeljka.

\chapter{Poglavlje 2}
Tekst 2. poglavlja.

\section{Odjeljak 2.1}
Tekst odjeljka 2.1.
\end{verbatim}

Tekst unutar poglavlja i odjeljaka dijeli se na paragrafe. Novi paragraf se
stvara preskakanjem jednog reda, primjerice:
\begin{verbatim}
It said: ``The History of every major Galactic Civilization tends
to pass through three distinct and recognizable phases, those of
Survival, Inquiry and Sophistication, otherwise known as the How,
Why and Where phases.''

``For instance, the first phase is characterized by the question
\emph{How can we eat?} the second by the question \emph{Why do we
eat?} and the third by the question \emph{Where shall we have
lunch?}''
  
  -- The Hitchhiker's Guide to the Galaxy by Douglas Adams
\end{verbatim}
Rezultat:

It said: ``The History of every major Galactic Civilization tends
to pass through three distinct and recognizable phases, those of
Survival, Inquiry and Sophistication, otherwise known as the How,
Why and Where phases.''

``For instance, the first phase is characterized by the question
\emph{How can we eat?} the second by the question \emph{Why do we
eat?} and the third by the question \emph{Where shall we have
lunch?}''
  
  -- The Hitchhiker's Guide to the Galaxy by Douglas Adams

Novi red u kodu ne čini novi red u dokumentu te prazni red stvara novi
paragraf. Da bi se dobio novi red bez stvaranja novog paragrafa, morate
iskoristiti sekvencu ``\verb|\\|'', primjerice (haiku Mjesec, Vladimir
Devid\'e):
\begin{verbatim}
Mirno jezero.\\
Žaba skoči na Mjesec\\
i sveg ga smrska.
\end{verbatim}
\boxresult{
Mirno jezero.\\
Žaba skoči na Mjesec\\
i sveg ga smrska.
}

Sekvencom ``\verb|\\|'' se ne može stvoriti prazna linija. Za prazninu između
redaka iskoristite naredbu \verb|\vspace{}|, npr., \verb|\vspace{10pt}|
(umjesto ``pt'' možete staviti drugu jedinicu).

\LaTeX{} također ignorira višestruke razmake, npr., ``\verb|tri   razmaka|''
rezultira s ``tri   razmaka.'' Dodatni razmak možete postići naredbom
\verb|\hspace{}|.

\subsection{Naredbe i okoline}
Izgled, stil i struktura \LaTeX{} dokumenta definiraju se naredbama i
okolinama. Naredbe su jednostavni elementi \LaTeX--a koji primaju argumente te
se transformiraju u oblikovani tekst. Naredbe počinju unazadnom kosom crtom
``\verb|\|'' koju slijedi naziv naredbe te njeni argumenti. Primjeri poziva
naredbe bez argumenta, s jednim argumentom te s dva argumenta:
\begin{verbatim}
\naredba
\naredba{a}
\naredba{a}{b}
\end{verbatim}
Uz argumente postoje i parametri. Argumenti naredbi se zadaju u vitičastim
zagradama, a parametri u uglatim (``[]'').

Okoline se mogu promatrati kao naredbe koje primaju jedan segment dokumenta.
Početak djelovanja okoline označava se naredbom \verb|\begin{}|, a kraj
naredbom \verb|\end{}|, primjerice:
\begin{verbatim}
\begin{okolina}
Segment dokumenta unutar okoline.
\end{okolina}
\end{verbatim}
Dodatni argumenti i parametri okoline navode se uz naredbu \verb|\begin{}|.

Neke često korištene \LaTeX{} naredbe se mogu vidjeti u tablici
\ref{tbl:ceste-naredbe}.
\begin{table}[htb]
\caption{Često korištene naredbe}
\label{tbl:ceste-naredbe}
\centering
\begin{tabular}{l l l l}
\toprule
Naredba & Opis & Primjer & Rezultat\\ \midrule
\texttt{emph} & Naglašavanje teksta & \verb|\emph{riječ}| & \emph{riječ} \\
\texttt{textbf} & Podebljavanje teksta & \verb|\textbf{riječ}| & \textbf{riječ} \\
\texttt{texttt} & ``Typewriter'' font & \verb|\texttt{riječ}| & \texttt{riječ}\\
\texttt{textsf} & ``Sans serif'' font & \verb|\textsf{riječ}| &  \textsf{riječ}\\
\texttt{ref} & Referenciranje & \verb|\ref{tbl:ceste-naredbe}| &  \ref{tbl:ceste-naredbe}\\
\texttt{url} & Formatiranje URL-ova & \verb|\url{www.hr}| &  \url{www.hr}\\
\bottomrule
\end{tabular}
\end{table}
% TODO: Dopuniti.
 
\subsection{Posebnosti \LaTeX-a i neki detalji o formatiranju}
Zbog tipografskih pravila \LaTeX{} ima posebnosti koje morate
poštivati da bi vaši dokumenti izgledali što ljepše.

Osnovni problem na koji ćete nailaziti proizlazi iz \LaTeX-ovih pokušaja
da poravna sadržaj. Pri poravnanju koristi se dodavanjem razmaka između riječi
i prelamanjem riječi na slogove. Ponekad se dogodi da je riječ loše prelomljena, 
primjerice, kad je potrebno prelomiti neku riječ koja sadrži dijakritike ili kad
koristite verbatim naredbe (uključujući \verb|\url{}| naredbu). Ukoliko dođe do
problema s prelamanjem, iskoristite naredbu \verb|\hyphenation{}| čiji primjer
uporabe možete vidjeti na \url{http://ahyco.ffri.hr/seminari2007/latex/2.1.html}.

Prema tipografskim pravilima nakon kraja rečenice (tj.\ točke) trebalo bi doći malo
više od standardnog razmaka. Zbog toga \LaTeX{} pretpostavlja da je svaka točka
kraj rečenice i tu ubaci malo veći razmak. Da bi to spriječili, možete iskoristiti
sekvencu ``\verb|\ |'' koja eksplicitno kaže da želite standardni razmak. Uz tu
sekvencu, bitan je i posebni znak ``\textasciitilde{}'' koji stvara standardni razmak
uz uvjet da se na tom razmaku linija teksta ne smije prelomiti, primjerice,  ako napišete
``\verb|prof.~dr.~sc.~Donald~E.~Knuth|'' \LaTeX{} vam garantira da će se cijelo ime s punom
titulom naći na istoj liniji. Naredbu ``\verb|\ |'' koristite uvijek nakon točke koja ne
označava kraj rečenice osim kad prethodni dio i dio nakon točke ne čine jednu neprelomivu
cjelinu, npr., titulu. Primjeri:
\begin{verbatim}
prof. dr. sc. Donald E. Knuth\\
prof.\ dr.\ sc.\ Donald E.\ Knuth\\
prof.~dr.~sc.~Donald~E.~Knuth
\end{verbatim}
\boxresult{prof. dr. sc. Donald E. Knuth\\
prof.\ dr.\ sc.\ Donald E.\ Knuth\\
prof.~dr.~sc.~Donald~E.~Knuth}

Ako pogledate dokumente napravljene \LaTeX-om, možete primijetiti različite vrste navodnika
među kojima svi dijele istu osobinu, razlikuje se otvaranje i zatvaranje navodnika. Budući
da se radi o prirodnom jeziku, računalu nije lako pretpostaviti kad se navodnik otvara, a kad
zatvara (prisjetite se samo problema s krajem rečenice), stoga morate koristiti posebne sekvence
ili naredbe da bi postigli ispravne navodnike. Preporuka je koristiti
\textasciigrave\textasciigrave \texttt{riječ}\textquotesingle\textquotesingle\ 
ili \verb|\glqq riječ\grqq|, a za navodnike unutar navodnika \textasciigrave \texttt{riječ}\textquotesingle.
Primjeri korištenja navodnika (citat, doc.~dr.~sc.~Siniša Šegvić):
\begin{verbatim}
"Bugovi su socijalna bića, uvijek dolaze u grupama."\\
`Bugovi su socijalna bića, uvijek dolaze u grupama.'\\
``Bugovi su socijalna bića, uvijek dolaze u grupama.''\\
\glqq Bugovi su socijalna bića, uvijek dolaze u grupama.\grqq
\end{verbatim}
\boxresult{"Bugovi su socijalna bića, uvijek dolaze u grupama."\\
`Bugovi su socijalna bića, uvijek dolaze u grupama.'\\
``Bugovi su socijalna bića, uvijek dolaze u grupama.''\\
\glqq Bugovi su socijalna bića, uvijek dolaze u grupama.\grqq}
Primjetite da se navodnici (kao i fusnote) pišu nakon točke.

Za stilski kvalitetan rad također je potrebno ispravno koristi crtice i trotočku.
U \LaTeX-u postoji više tipova crtica, kratka, ``-'', srednja ``--'' i duga
``---'' te posebna naredba, \verb|\ldots| koja ispravno prikazuje trotočku (štoviše, ponekad vam
i MS Word pokuša produžiti crticu ako je dovoljno očito ono što želite postići te
skoro uvijek pokuša ispravno prikazati trotočku). Kratku crticu (\verb|-|) koristite između riječi,
srednju (\verb|--|) između brojeva, a dugu (\verb|---|) pri odvajanju podrečenice. U našem jeziku
korištenje crtice nije strogo definirano,\footnote{Više o crticama pročitajte u članku: \url{http://hrcak.srce.hr/file/67827}}
stoga su dane upute samo prijedlog. Ako trebate ``minus'' ispred broja, uvijek uđite u mod za matematiku.
Primjeri za crtice i trotočku:
\begin{verbatim}
``agencija `Drugdje'...''\\
``agencija `Drugdje'\ldots''\\
``hrvatsko-njemački''\\
``31--42''\\
``rekao je `može' --- uglavnom iz straha --- te nastavio''\\
``to je $-1$, a ne -1''
\end{verbatim}
\boxresult{``agencija `Drugdje'...''\\
``agencija `Drugdje'\ldots''\\
``hrvatsko-njemački''\\
``31--42''\\
``rekao je `može' --- uglavnom iz straha --- te nastavio''\\
``to je $-1$, a ne -1''}

\LaTeX{} sadrži posebne znakove koji su osnova njegove sintakse. Ako te znakove koristite u običnom tekstu,
morate ih \emph{isključiti} \engl{escape} znakom ``\verb|\|.'' Popis posebnih znakova i njihov prikaz u teksu:
\begin{verbatim}
\$  \%  \_  \}  \{  \&  \#
\end{verbatim}
\boxresult{\$  \%  \_  \}  \{  \&  \#}
Znak ``\textbackslash'' se ne može dobiti isključivanjem, već morate iskoristiti naredbu iz standardnog
\LaTeX{} paketa, \verb|\textbackslash|.

Svi navedeni znakovi osim \# i \% se mogu susresti u naredbama unutar ovih uputa. Znak \# služi
za naprednije \TeX{} programiranje (referenciranje argumenata), a \% kao početak linijskog komentara, primjerice:
\begin{verbatim}
Film ``\$9.99'' je odličan! % TODO: Ponovo pogledati.
\end{verbatim}
\boxresult{
Film ``\$9.99'' je odličan! % TODO: Ponovo pogledati.
}

\section{Dodavanje popisa sadržaja, slika i tablica}
Sadržaj se dodaje naredbom \verb|\tableofcontents| na mjestu na kojemu ga želite
prikazati. Analogno tome, popis slika se dodaje naredbom \verb|\listoffigures|,
a popis tablica naredbom \verb|\listoftables|.

\section{Dodavanje fusnota i referenci}
Fusnote se dodaju naredbom \verb|\footnote{}|, npr.:
\begin{verbatim}
Popis fusnota\footnote{Objašnjenja koja se prikazuju
na dnu stranice} nije potreban.
\end{verbatim}
\boxresult{Popis fusnota\footnote{Objašnjenja koja se
prikazuju na dnu stranice} nije potreban.}

Reference služe za povezivanje s nekim dijelom rada. Da biste omogućili
referenciranje na neki dio rada, tom dijelu morate postaviti oznaku
\engl{label} naredbom \verb|\label{}|, npr.:
\begin{verbatim}
\section{Dodavanje posvete}
\label{sec:posveta}
\end{verbatim}
referencirate se pomoću naredbe \verb|\ref{}|, npr.:
\begin{verbatim}
Za navedeno pogledajte odjeljke \ref{sec:sazetak},
\ref{sec:posveta} i \ref{sec:engl}
\end{verbatim}

Oznaku za referenciranje možete postaviti većini elemenata, a najčešće se
koristi za označavanje poglavlja, odjeljaka, jednadžbi, tablica i slika.


\section{Dodavanje listi, nabrajanja i opisa}
Liste tvore u okolini \texttt{itemize}, nabrajanja u okolini
\texttt{enumerate}, a opisi u okolini \texttt{description}. Primjeri:

\begin{description}
  \item[Obična lista] -- kod:
\begin{verbatim}
\begin{itemize}
  \item prva stavka,
  \item druga stavka.
\end{itemize}
\end{verbatim}
\boxresult{
\begin{itemize}
  \item prva stavka,
  \item druga stavka.
\end{itemize}
}

  \item[Lista s više razina] -- kod:
\begin{verbatim}
\begin{itemize}
  \item prva stavka,
  \item druga stavka,
  \begin{itemize}
    \item druga razina. 
  \end{itemize}
\end{itemize}
\end{verbatim}
\boxresult{
\begin{itemize}
  \item prva stavka,
  \item druga stavka,
  \begin{itemize}
    \item druga razina. 
  \end{itemize}
\end{itemize}}

  \item[Nabrajanje] -- kod:
\begin{verbatim}
\begin{enumerate}
  \item prva stavka,
  \item druga stavka.
\end{enumerate}
\end{verbatim}
\boxresult{
\begin{enumerate}
  \item prva stavka,
  \item druga stavka.
\end{enumerate}}

  \item[Nabrajanje s proizvoljnim brojačem] -- kod:
\begin{verbatim}
\begin{enumerate}[(a)]
  \item prva stavka,
  \item druga stavka.
\end{enumerate}
\end{verbatim}
\boxresult{
\begin{enumerate}[(a)]
  \item prva stavka,
  \item druga stavka.
\end{enumerate}}

  \item[Opisi] -- kod:
\begin{verbatim}
\begin{description}
  \item[Esperanto:] najpoznatiji umjetni prirodni jezik.
  \item[Lojban:] sintaksno jednoznačan umjetni prirodni jezik.
  \item[Toki pona:] minimalistički umjetni prirodni jezik.
\end{description}
\end{verbatim}
\boxresult{
\begin{description}
  \item[Esperanto:] najpoznatiji umjetni prirodni jezik.
  \item[Lojban:] sintaksno jednoznačan umjetni prirodni jezik.
  \item[Toki pona:] minimalistički umjetni prirodni jezik.
\end{description}}

\end{description}

Za izradu složenijih listi, nabrajanja i opisa iskoristite paket
\texttt{enumitem}. Odličan prikaz njegovih mogućnosti iznosi
\citep{collins2008enum}.


\section{Dodavanje tablica}
\label{sec:tablice}
Tablice dodajete preko \texttt{tabular} i \texttt{table} okolina. Primjer
tablice:
\begin{verbatim}
\begin{table}[htb]
\caption{Konstante}
\label{tbl:konstante}
\centering
\begin{tabular}{llr} \toprule
Konstanta & Opis & Vrijednost\\ \midrule
$\pi$ & Pi & 3.14159 \\
$e$ & Eulerov broj & 2.71828 \\
$\varphi$ & Zlatni rez & 1.61803 \\ \bottomrule
\end{tabular}
\end{table}
\end{verbatim}
Rezultat je tablica \ref{tbl:konstante}.
\begin{table}[htb]
\caption{Konstante}
\label{tbl:konstante}
\centering
\begin{tabular}{llr} \toprule
Konstanta & Opis & Vrijednost\\ \midrule
$\pi$ & Pi & 3.14159 \\
$e$ & Eulerov broj & 2.71828 \\
$\varphi$ & Zlatni rez & 1.61803 \\ \bottomrule
\end{tabular}
\end{table}

Opis danog primjera po dijelovima:
\begin{itemize}
\item \verb|\begin{table}[htb]|\\*
Početak \texttt{table} okoline s pozicioniranjem, ``h'' -- \emph{here}, ``t'' --
\emph{top}, ``b'' -- \emph{bottom}. Table je tzv.\ lebdeća \engl{floating}
okolina te joj pozicioniranjem govorimo gdje želimo da se naša tablica postavi.
\LaTeX{} ima svoje algoritme pozicioniranja lebdećih okolina tako da ih smješta
po tipografskim pravilima npr., \LaTeX{} neće dopustiti da se tablica razvuče na
dvije stranice tako da se prelomi po sredini, već će cijelu tablicu prebaciti na
novu stranicu. Ako se nađete u situaciji da vam pozicija gdje je \LaTeX{}
smjestio lebdeću okolinu ne odgovara, kao parametar pozicioniranja možete navesti
``!'' (npr., ``[!htb]''), no preporuka je pozicioniranje prepustiti \LaTeX-ovim
algoritmima. Bitno je spomenuti da se za tablice koje veličinom prelaze jedan
list papira mora koristiti \texttt{longtable} paket. Paket je uključen u predložak, a
upoznavanje s paketom je najbolje napraviti u primjerima koje možete naći na
\url{http://users.sdsc.edu/~ssmallen/latex/longtable.html} ili
\url{http://www.astro.psu.edu/gradinfo/psuthesis/longtable.html}.

\item \verb|\caption{Konstante}|\\*
Naslov tablice. Naslov tablice uvijek dolazi iznad tablice.

\item \verb|\label{tbl:konstante}|\\*
Oznaka preko koje se referencirate na tablicu npr., \verb|\ref{tbl:konstante}|.

\item \verb|\centering|\\*
Naredba koja centrira tablicu.

\item \verb|\begin{tabular}{llr}|\\*
Okolina za unos tabličnih podataka. Parametar ``\verb|{llr}|'' navodi da tablica
ima tri stupca, pri čemu su prva dva poravnata lijevo, a treći je poravnat
desno. Ako želite vertikalnu liniju između prvog i drugog stupca, kao parametar
morate navesti ``\verb@{l|lr}@.'' Dobar pregled parametara i njihovih
objašnjenja možete naći na
\url{http://www.andy-roberts.net/misc/latex/latextutorial4.html}.

\item 
\begin{verbatim}
Konstanta & Opis & Vrijednost\\ \midrule
$\pi$ & Pi & 3.14159 \\
$e$ & Eulerov broj & 2.71828 \\
$\varphi$ & Zlatni rez & 1.61803 \\ \bottomrule
\end{verbatim}
Podatci u tablici. Podatci u pojedinim stupcima se odvajaju znakom ``\&''. Redovi
tablice prekidaju se sekvencom ``\verb|\\|.'' Naredbe \verb|\toprule|,
\verb|\midrule|, \verb|\bottomrule| i \verb|\hline| odvajaju retke horizontalnim
linijama. Same naredbe razlikuju se u debljini linije koju stvaraju i količinama
praznine iznad i ispod linije. Preporuka je raditi tablice bez vertikalnih
linija, s linijom \verb|\toprule| iznad naslova stupaca, linijom \verb|\midrule|
ispod naslova te linijom \verb|\bottomrule| ispod zadnjeg podatka
\citep{simon2005booktabs}.
\end{itemize}

Prilikom izrade tablica preporuka je poštivati smjernice dane u
dokumentaciji \texttt{booktabs} paketa \citep{simon2005booktabs}:
\begin{itemize}
  \item nemojte koristiti vertikalne linije,
  \item nemojte koristiti dvostruke linije,
  \item jedinice stavljajte u naslov stupca, ne u tijelo tablice\\*
(npr., ako imate stupac ``Masa'', umjesto da za podatke navodite npr., ``1
kg'', jedinicu stavite u naslov -- ``Masa (kg)''),
  \item uvijek stavljajte broj ispred decimalne točke; znači ``0.1,'' a ne
``.1,''
  \item nemojte koristiti ``ditto'' znak za ponavljanje prethodne vrijednosti;
ostavite praznu liniju ili ponovo navedite vrijednost.
\end{itemize}

\section{Dodavanje slika}
Slike se dodaju pomoću \texttt{figure} okoline. Primjer je slika
\ref{fig:fer-logo}. Kod kojim je dobivena slika:
\begin{verbatim}
\begin{figure}[htb]
\centering
\includegraphics[width=2cm]{img/FER_logo.jpg}
\caption{Logo FER-a}
\label{fig:fer-logo}
\end{figure}
\end{verbatim}

Okolina \texttt{figure}, poput \texttt{table} okoline, spada u lebdeće okoline i
ne mora se naći točno na mjestu na kojem ste je naveli. Za detalje pogledajte
odjeljak \ref{sec:tablice}

\begin{figure}[htb]
\centering
\includegraphics[width=1.7cm]{img/FER_logo.jpg}
\caption{Logo FER-a.}
\label{fig:fer-logo}
\end{figure}

Naredba \verb|\includegraphics{}| prima slike u \emph{jpg}, \emph{png} i
\emph{pdf} formatima ako se koristi s \emph{pdflatex} prevoditeljem. Dodatnim
opcijama naredbe sliku je moguće skalirati po visini (npr.,
\texttt{[height=2cm]}), širini ili za određeni faktor (parametar
\texttt{scale}) te rotirati za određeni kut u stupnjevima (parametar
\texttt{angle}). Detaljnije o uključivanju slika u \LaTeX{} dokumente
pročitajte na \url{http://en.wikibooks.org/wiki/LaTeX/Importing_Graphics}.
Alati poput \emph{Mathematice} mogu generirati slike u
vektorskom formatu \emph{eps} bez definiranog ``bounding boxa.'' Za ispravno
definiranje ``bounding boxa'' vektorskih formata možete iskoristiti
besplatni, višeplatformski alat za vektorsku grafiku \emph{inkscape} dostupan na
\url{http://www.inkscape.org/} ili konzolni alat \emph{ebb} koji dolazi uz
\LaTeX{} distribucije te generira ``bounding box'' i za rasterske formate poput
\emph{png} formata. Primjer korištenja \emph{ebb} alata za sliku ``slika.png'':
\begin{verbatim}
ebb slika.png
\end{verbatim}
Alat generira datoteku ``slika.bb'' koju stavite u direktorij sa slikom.

Ako sliku trebate staviti unutar teksta, koristite paket \texttt{wrapfig}, a
za prikaz više slika u nizu koristite paket \texttt{subfig}. Oba paketa su
uključena u predložak. Za primjere korištenja
\texttt{wrapfig} i \texttt{subfig} paketa te savjete oko korištenja lebdećih
okolina pogledajte stranicu
\url{http://en.wikibooks.org/wiki/LaTeX/Floats,_Figures_and_Captions}.


\section{Dodavanje matematičkih izraza}
Matematički izrazi se pišu u posebnoj okolini u koju se ulazi s \verb|$|
\ldots \verb|$|, \verb|\(| \ldots \verb|\)| za linijske izraze te s \verb|\[| \ldots
\verb|\]| ako želite da se ti izrazi nalaze u posebnoj liniji. Primjerice, za
linijske izraze:
\begin{verbatim}
``Prva dama: $\sin^2 \varphi + \cos^2 \varphi = 1$.''
\end{verbatim}
\boxresult{``Prva dama: $\sin^2 \varphi + \cos^2 \varphi = 1$.''}
Za izraze koji se nalaze u posebnoj liniji:
\begin{verbatim}
\[ c^2 = a^2 + b^2 \]
\end{verbatim}
\boxresult{
\[ c^2 = a^2 + b^2 \]
}

U matematičku okolinu za izraze koji se nalaze u posebnoj liniji moguće je ući
s \verb|$$| \ldots \verb|$$|, ali nije preporučljivo jer nije kompatibilno s
nekim mogućnostima \LaTeX-a (no, unatoč tomu, često se koristi).

Također, postoji još okolina za pisanje matematičkih izraza među kojima je
bitno istaknuti okoline \texttt{equation}, koja donosi mogućnost označavanja
jednadžbi (oznake se automatski generiraju), i \texttt{align} koja omogućava
pisanje jednadžbi u više redova. Primjer za \texttt{equation} okolinu:
\begin{verbatim}
\begin{equation}
f(t)\ast g(t) = \int^{\infty}_{-\infty} f(\tau)g(t-\tau)d\tau.
\label{eq:conv}
\end{equation}
\end{verbatim}
\boxresult{
\begin{equation}
f(t)\ast g(t) = \int^{\infty}_{-\infty} f(\tau)g(t-\tau)d\tau.
\label{eq:conv}
\end{equation}}
Primjer za \texttt{align} okolinu:
\begin{verbatim}
\begin{align}
a&=b+c,\label{eq:a}\\
d&=e+f+g,\\
h&=i+j.\label{eq:h}
\end{align}
\end{verbatim}
\boxresult{
\begin{align}
a&=b+c,\label{eq:a}\\
d&=e+f+g,\\
h&=i+j.\label{eq:h}
\end{align}}

Osim uzastopnog prikaza više jednadžbi i poravnanja po znaku jednakosti, primjer
pokazuje da je moguće označiti svaku pojedinu jednadžbu. Primjerice:
\begin{verbatim}
``izraz za $h$ je dan jednadžbom \ref{eq:h}''
\end{verbatim}
\boxresult{``izraz za $h$ je dan jednadžbom \ref{eq:h}''}


Za dodatne informacije o pisanju matematičkih izraza u \LaTeX-u pogledajte
web stranice \url{http://www.math.uiuc.edu/~hildebr/tex/displays.html} i
\url{http://www.andy-roberts.net/misc/latex/latextutorial10.html} te pročitajte
kratke upute za korištenje AMS paketa
\url{ftp://ftp.ams.org/pub/tex/doc/amsmath/short-math-guide.pdf}. AMS paket je
uključen u predložak. \nocite{downes2002shortams}

\section{Dodavanje programskog koda, sadržaja koji mora ostati neformatiran i
pseudokoda}
Za programski kod možete iskoristiti paket \texttt{listings}. Paket
nije uključen u predložak te ga morate dodati kako je opisano u odjeljku
\ref{sec:koristenje-dod-paketa}. Paket je potrebno dodatno konfigurirati da
odgovara korištenom jeziku. Prije korištenja paketa dodajte naredbu, npr.\ za
programski jezik Javu:
\begin{verbatim}
\lstset{language=Java, tabsize=2}
\end{verbatim}
Primjer:
\begin{lstlisting}
public class TempIdentificatorFactory {
	/** Pocetna brojcana vrijednost. */
	private static int num = 0;
	
	/**
	 * Generiranje unikatnih identifikatora
	 * privremenih varijabli.
	 * @return novi unikatni identifikator.
	 */
	public static String generateIdentificator() {
		String newIdn = new String(num + "_tmp");
		num++;
		return newIdn;
	}
}
\end{lstlisting}
Kod je stavljen unutar okoline \texttt{lstlisting}:\\
\verb|\begin{lstlisting}|\\*
\verb|Kod.|\\*
\verb|\end{lstlisting}|

Paket \texttt{listings} ne podržava UTF-8 kodnu stranicu te za korištenje
dijakritika morate iskoristiti neko drugo rješenje, npr.\ paket
\texttt{listingsutf8} ili okolinu \texttt{verbatim} koja spriječava
formatiranje.

Navedena okolina ima širu primjenu, tj.\ služi za sav sadržaj za koji želite da
održi razmake i prelaske u novi red iz izvorne \texttt{tex} datoteke, te da se
\TeX{} naredbe unutar tog sadržaja ne bi izvršile. Primjer korištenja:\\
\verb|\begin{verbatim}|\\*
\verb|Neki      tekst.|\\*
\verb|\end{verbatim}|\\
rezultat:
\begin{verbatim}
Neki      tekst.
\end{verbatim}

Okolina \texttt{verbatim} ima svoj linijski ekvivalent, \texttt{verb} naredbu.
Primjer korištenja:
\begin{verbatim}
Tekst bez naglaska \verb|\emph{ove riječi}|.
\end{verbatim}
Rezultat: ``Tekst bez naglaska \verb|\emph{ove riječi}|.''

Naredba \texttt{verb} funkcionira na način da ostavlja neobrađenim sve od
prve pojave odjeljitelja, koji se određuje kao prvi znak nakon same naredbe (u
primjeru to je znak ``|''), do njegove druge pojave.\footnote{Sličnu ideju s
delimiterima koristi \texttt{sed} naredba Unix ljuski.}

Pseudokod, osim u obliku neformatiranog teksta, možete dodavati pomoći više
različitih paketa koje morate posebno uključiti. Primjer takvih paketa su
\texttt{algorithmic} i \texttt{algorithm}. Navedene pakete morate dodati na način
koji je opisan u \ref{sec:koristenje-dod-paketa}. Primjer korištenja je dan
algoritmom~\ref{algo:bubble-sort} (radi se o lebdećoj okolini koja se ne mora
naći točno gdje je napisana). \LaTeX{} kod algoritma:
\begin{verbatim}
\begin{algorithm}
\caption{Bubble sort}
\label{algo:bubble-sort}
\begin{algorithmic}
\STATE{\textbf{Ulaz:} $A$ -- niz koji treba sortirati.}
\STATE{\textbf{Izlaz:} sortirani niz.}
\REPEAT
\STATE{swapped := false}
\STATE{n := length(A)}
\FOR{($i := 0; i < n; inc(i)$)}
\IF{$A_{i} > A_{i+1}$}
\STATE{$swap(A_i, A_{i + 1})$}
\STATE{swapped := true}
\ENDIF
\ENDFOR
\STATE{$n := n - 1$}
\UNTIL{$\lnot swapped$}
\RETURN{A}
\end{algorithmic}
\end{algorithm}
\end{verbatim}

\begin{algorithm}
\caption{Bubble sort}
\label{algo:bubble-sort}
\begin{algorithmic}
\STATE{\textbf{Ulaz:} $A$ -- niz koji treba sortirati.}
\STATE{\textbf{Izlaz:} sortirani niz.}
\REPEAT
\STATE{swapped := false}
\STATE{n := length(A)}
\FOR{($i := 0; i < n; inc(i)$)}
\IF{$A_{i} > A_{i+1}$}
\STATE{$swap(A_i, A_{i + 1})$}
\STATE{swapped := true}
\ENDIF
\ENDFOR
\STATE{$n := n - 1$}
\UNTIL{$\lnot swapped$}
\RETURN{A}
\end{algorithmic}
\end{algorithm}


\section{Podjela sadržaja na više stupaca}
Za podjelu sadržaja na stupce koristite \texttt{multicol} paket koji je
uključen u predložak. Primjer korištenja:\footnote{Korišteni citati pripadaju
Donaldu E.~Knuthu, tvorcu \TeX-a}
\begin{verbatim}
\begin{multicols}{2}
The most important thing in the programming language is the
name. A language will not succeed without a good name. I have
recently invented a very good name and now I am looking for a
suitable language.

The hardest thing is to go to sleep at night, when there are
so many urgent things needing to be done. A huge gap exists
between what we know is possible with today's machines and
what we have so far been able to finish.
\end{multicols}
\end{verbatim}
\boxresult{
\begin{multicols}{2}
The most important thing in the programming language is the name. A language will
not succeed without a good name. I have recently invented a very good name and
now I am looking for a suitable language.

The hardest thing is to go to sleep at night, when there are so many urgent
things needing to be done. A huge gap exists between what we know is possible
with today's machines and what we have so far been able to finish.
\end{multicols}}

Bitno je primijetiti da \texttt{multicol} okolina nije kompatibilna s
\texttt{figure} i \texttt{table} okolinama. Za ubacivanje \texttt{figure}
okoline unutar \texttt{multicol} okoline potrebno je koristiti
\texttt{multipage} paket, primjerice:
\begin{verbatim}
\begin{minipage}{\linewidth} 
\vspace{10pt}
\centering% 
\includegraphics[width=0.8\linewidth]{sample-fig.jpg}% 
\figcaption{Slika unutar multicol okoline}% 
\label{fig:sample-fig}% 
\end{minipage}
\end{verbatim}

Za korištenje tablica iskoristite također \texttt{multipage} paket sa
\texttt{tabular} okolinom ili, ako imate šire tablice za koje želite da se pojave
u širini cijele stranice, \texttt{table*} okolinu koja ima funkcionalnost
\texttt{table} okoline, ali se uvijek proteže cijelom širinom stranice.

\section{Dodavanje literature}
\label{sec:literatura}
Za navođenje literature u \LaTeX{} dokumentima zadužen je alat Bib\TeX{}.
Bib\TeX{} omogućuje korištenje raznih stilova prikaza navoda literature i
referenciranja preko naredbe \verb|\bibliographystyle{}|.

Literaturu možete navoditi unutar \texttt{thebibliography} okoline ili
korištenjem \texttt{bib} baze literature (preporučeni način). Baza literature je
standardna tekstualna datoteka sa nastavkom ``bib'' koja sadrži Bib\TeX{} zapise,
a u dokument se uključuje naredbom \verb|\bibliography{}|, npr.\
\verb|\bibliography{literatura}| za datoteku \texttt{literatura.bib}. Primjer
Bib\TeX{} zapisa:
\begin{verbatim}
@article{greenwade93,
  author  = "George D. Greenwade",
  title   = "The {C}omprehensive {T}ex {A}rchive {N}etwork",
  year    = "1993",
  journal = "TUGBoat",
  volume  = "14",
  number  = "3",
  pages   = "342--351"
}
\end{verbatim}
Tip zapisa \emph{article} navodi da se radi o članku. Neki drugi tipovi zapisa
su \emph{book}, \emph{manual}, \emph{techreport}, \emph{misc}, i sl. Razlikuju
se u propisanim obaveznim i neobaveznim poljima te semantici (neki Bib\TeX{}
stilovi različito prikazuju pojedine tipove zapisa).

Pretraživači baza znanstvenih članaka poput Google
Scholara\footnote{\url{http://scholar.google.hr/}} i
Citeseera\footnote{\url{http://citeseerx.ist.psu.edu/}} uz svaki članak nude i
njegov Bib\TeX{} zapis. Za Google Scholar tu opciju je potrebno eksplicitno
omogućiti (u postavkama navedite da se prikazuju veze za uvoz navoda iz
Bib\TeX-a).


Za referenciranje na navod literature koriste se naredbe \verb|\citep{}| i
\verb|citet{}| (ove naredbe su specifične za \texttt{natbib} paket koji se
koristi u ovom predlošku, sam Bib\TeX{} koristi \verb|\cite{}| naredbu). Kao
parametar tim naredbama predaje se ključ navoda, npr.\ ``greenwade93'' iz
gornjeg primjera. Naredba \texttt{citep} proizvodi standardni navod literature,
a \texttt{citet} navodi ime autora, primjerice:
\begin{verbatim}
Primjer korištenja enumitem paketa dan je u \citep{collins2008enum}.
\citet{collins2008enum} opisuje korištenje enumitem paketa.
\end{verbatim}

\boxresult{``Primjer korištenja enumitem paketa dan je u
\citep{collins2008enum}.''\\*
``\citet{collins2008enum} opisuje korištenje enumitem paketa.''}

Više o \texttt{natbib} paketu možete pročitati na
\url{http://merkel.zoneo.net/Latex/natbib.php}, a više detalja o dodavanju
literature u \LaTeX{} dokumente možete naći na
\url{http://en.wikibooks.org/wiki/LaTeX/Bibliography_Management}.

\subsection{Promjena načina citiranja}
Trenutni način citiranja je ``(autor, godina).'' Ako želite promijeniti stil
citiranja u ``[indeks]'' (npr.\ [1]) tada u datoteci \texttt{zavrsni.cls}
promijenite parametre \emph{natbib} paketa:
\begin{verbatim}
\RequirePackage[authoryear, round]{natbib}
\end{verbatim}
u
\begin{verbatim}
\RequirePackage[numbers, square]{natbib}
\end{verbatim}

% \section{Dodavanje indeksa}
% Ma, ovo je nepotrebno..
% http://www.tex.ac.uk/cgi-bin/texfaq2html?label=makeindex

\section{Dodatci dokumenta \engl{appendix}}
Dodavanje dodataka vrši se dodavanjem poglavlja nakon naredbe \verb|\appendix|.
Ta poglavlja se označavaju velikim latiničnim slovima. Naredba \verb|\appendix|
dolazi nakon literature (vidi odjeljak \ref{sec:literatura}). Primjer:
\begin{verbatim}
\appendix
\chapter{Korisne web stranice o \LaTeX-u}
\end{verbatim}
\boxresult{``Dodatak A\\*
Korisne web stranice o \LaTeX-u''}


\section{Dodavanje sažetka rada}
\label{sec:sazetak}
Sažetak rada je obavezan dio koji dolazi na sam kraj rada. Za sažetak na
hrvatskom jeziku koristite okolinu \emph{sazetak}, a za
sažetak na engleskom jeziku okolinu \emph{abstract}. Prije korištenja
\emph{abstract} okoline potrebno je navesti naslov na engleskom jeziku naredbom
\verb|\engtitle{}|. Primjer korištenja:
\begin{verbatim}
\begin{sazetak}
Sažetak rada.

\kljucnerijeci{<popis ključnih riječi>}
\end{sazetak}

\engtitle{Naslov rada na engleskom jeziku}
\begin{abstract}
Sažetak na engleskom jeziku.

\keywords{<popis ključnih riječi na engleskom>}
\end{abstract}
\end{verbatim}
Nakon sažetka dolazi kraj dokumenta, odnosno naredba:
\begin{verbatim}
\end{document}
\end{verbatim}

U popis ključnih riječi navedite pojmove bitne za vaš rad koji nisu već
spomenuti u naslovu.

\section{Dodavanje posvete}
\label{sec:posveta}
Posveta se dodaje naredbom \verb|\posveta{}| na mjestu gdje želite stranicu s
posvetom (npr.\ prije tablice sadržaja -- iznad \verb|\tableofcontents|
naredbe). Primjer korištenja:
\begin{verbatim}
\posveta{Svima koji žele naučiti koristiti \LaTeX{}.}
\end{verbatim}


\section{Korištenje naredbe \texttt{engl}}
\label{sec:engl}
Naredba \verb|\engl{}| služi navođenju engleskog
prijevoda nekog termina (radi se o nestandardnoj naredbi), npr.:
\begin{verbatim}
Dodatak dokumenta \engl{appendix}.
\end{verbatim}
\boxresult{Dodatak dokumenta \engl{appendix}.}


\section{Korištenje dodatnih paketa}
\label{sec:koristenje-dod-paketa}
\LaTeX{} paketi se mogu promatrati kao dodatne biblioteke u programskim
jezicima. Dodavanje novih paketa vrši se \verb|\usepackage{}| naredbom.
Navedena naredba dolazi nakon \verb|\documentclass{}| naredbe. Primjer
korištenja:
\begin{verbatim}
\documentclass{zavrsni}
\usepackage{algorithmic}
\usepackage{algorithm}
\usepackage{listings}
\usepackage{longtable}
\end{verbatim}

\LaTeX{} distribucije dolaze s velikim brojem već instaliranih paketa.
Također, distribucije je moguće podesiti da, u slučaju korištenja paketa koji
nije instaliran, same dohvate paket sa CTAN-a i instaliraju ga.

Za više informacija o \LaTeX{} paketima, konzultirajte CTAN
(\url{http://tug.ctan.org/}). Korisne upute za korištenje i instalaciju \LaTeX{}
paketa možete naći na \url{http://en.wikibooks.org/wiki/LaTeX/Packages}.

Bitno je spomenuti da se uz svaki \TeX{} paket s CTAN-a može skinuti pripadna
dokumentacija.


\section{Provjera pravopisa}
Programi \emph{aspell} i \emph{ispell} omogućavaju jednostavnu provjeru
pravopisa \LaTeX{} dokumenata (razumiju sintaksu \TeX{}-a). Primjer korištenja:
\begin{verbatim}
aspell -c dokument.tex
\end{verbatim}
ili
\begin{verbatim}
ispell dokument.tex
\end{verbatim}
Za više informacija, instalaciju i podešavanje navedenih alata konzultirajte:
\begin{itemize}
  \item \url{http://aspell.net/}
  \item \url{http://www.gnu.org/software/ispell/ispell.html}
  \item \url{http://cvs.linux.hr/spell/}
  \item \url{http://aspell.net/win32/}
  \item \url{http://gustav.fesb.hr/hr/ispell.html}
  \item \url{http://en.wikipedia.org/wiki/GNU_Aspell}
\end{itemize}


\bibliography{literatura}
\bibliographystyle{plainnat}


\appendix
\chapter{Korisne web stranice o \LaTeX-u}
\label{ch:korisne-stranice}
% TODO: Ovo nekako ljepše složiti
% TODO: Sortirati po korisnosti i navesti kriterij sortiranja.
\textbf{LaTeX – A document preparation system:}
\begin{itemize}
  \item \url{http://www.latex-project.org/}
  \item Službena stranica \LaTeX{} projekta.
\end{itemize}
\textbf{The Comprehensive TeX Archive Network (CTAN):}
\begin{itemize}
  \item \url{http://www.ctan.org/}
  \item Osnovna arhiva paketa i materijala vezanih uz \TeX\ sustav.
\end{itemize}
\textbf{TeX Frequently Asked Questions on the Web:}
\begin{itemize}
  \item \url{http://www.tex.ac.uk/cgi-bin/texfaq2html}
  \item Veliki \TeX\ i \LaTeX{} FAQ sa konkretnim problemima i poveznicama na
  korisne stranice.
\end{itemize}
\textbf{The Not So Short Introduction to LATEX2e:}
\begin{itemize}
  \item \url{http://tobi.oetiker.ch/lshort/lshort.pdf}
  \item Izuzetno hvaljena knjiga o \LaTeX-u prigodna za početnike.
\end{itemize}
\textbf{Šime Ungar -- Ne baš tako kratak uvod u TeX s naglaskom na LaTeX2e:}
\begin{itemize}
  \item \url{http://web.math.hr/~ungar/lkratko2e_internet.pdf}
  \item Besplatna knjiga o \LaTeX-u na Hrvatskom jeziku.
\end{itemize}
\textbf{\LaTeX{} wikibook:}
\begin{itemize}
  \item \url{http://en.wikibooks.org/wiki/LaTeX}
  \item Wiki stranica sa objašnjenjima i kvalitetnim \LaTeX{} receptima.
\end{itemize}
\textbf{\LaTeX{} courseware:}
\begin{itemize}
  \item \url{http://ahyco.ffri.hr/seminari2007/latex/home.html}
  \item Odličan \emph{courseware} pod nazivom \LaTeX{} nastao u okviru kolegija
  ``Metodika nastave informatike II'' studijske grupe Matematika i informatika Filozofskog
  fakulteta u Rijeci.
\end{itemize}
\textbf{Detexify$^2$ -- \LaTeX{} symbol classifier:}
\begin{itemize}
  \item \url{http://detexify.kirelabs.org/classify.html}
  \item Interaktivni prepoznavatelj simbola. Iznimno koristan alat ako tražite
  neki simbol.
\end{itemize}
\textbf{IEEE predavanje -- Uvod u \LaTeX:}
\begin{itemize}
  \item \url{http://www.fer.hr/ieee?@=g4ct}
  \item Predavanje koje je u travnju 2007.\ godine održao mr.~sc.~Tomislav
  Petković. Popratni materijali koji se mogu naći u repozitoriju su izuzetno
  korisni.
\end{itemize}
\textbf{\TeX{}ample TikZ and PGF:}
\begin{itemize}
  \item \url{http://www.texample.net/tikz/}
  \item Primjeri korištenja \texttt{tikz} paketa za izradu izuzetno složenih
  dijagrama.
\end{itemize}
\textbf{Getting to grips with \LaTeX{}:}
\begin{itemize}
  \item \url{http://www.andy-roberts.net/misc/latex/}
  \item \LaTeX{} tutoriali s velikim brojem primjera i objašnjenjima raznih
  parametara često korištenih naredbi.
\end{itemize}
\textbf{Art of problem solving \LaTeX{} wiki:}
\begin{itemize}
  \item \url{http://www.artofproblemsolving.com/Wiki/index.php/LaTeX:About}
  \item \LaTeX{} wiki sa konkretnim primjerima.
\end{itemize}
\textbf{IMAGE Lab, University of Florida, College of Liberal Arts \& Sciences
\LaTeX{} overview:}
\begin{itemize}
  \item \url{http://www.image.ufl.edu/help/latex/}
  \item Primjeri i upute za korištenje \LaTeX{}-a temeljeni na knjigama \emph{A
  Beginner's Introduction to Typesetting with LaTeX} i \emph{The Not So Short
  Introduction to LaTeX2e}.
\end{itemize}
\textbf{The Comprehensive \LaTeX{} Symbol List}
\begin{itemize}
  \item
  \url{http://www.ctan.org/tex-archive/info/symbols/comprehensive/symbols-letter.pdf}
  \item Lista 5913 simbola dostupna iz raznih paketa na CTAN-u.
\end{itemize}
\textbf{The Visual \LaTeX{} FAQ}
\begin{itemize}
  \item \url{http://mirror.ctan.org/info/visualFAQ/visualFAQ.pdf}
  \item Izuzetno dobro napravljen skup \LaTeX{} recepata koji se temelji na
  konkretnom primjeru recepta i poveznicom na Internet stranicu sa detaljnim
  opisom recepta.
\end{itemize}
\textbf{\TeX{} by Topic}
\begin{itemize}
  \item \url{https://savannah.nongnu.org/projects/texbytopic}
  \item Odlična knjiga za napredne korisnike \TeX-a (elektronska verzija je besplatna).
\end{itemize}


\begin{sazetak}
Upute za korištenje razvijenih predložaka za Završni i Diplomski rad te Seminar
na Fakultetu elektrotehnike i računarstva. Upute se sastoje od dijela koji se
tiče samo predložaka i dijela koji se tiče \LaTeX-a općenito. Dio o \LaTeX-u
napravljen s ciljem da bude dovoljan kao uvod u rješavanje pojedinih problema
pri formatiranju, ali ne i preopširan.

\kljucnerijeci{FER, \TeX{}.}
\end{sazetak}

\engtitle{Manual for Bachelor thesis, Master thesis and Seminar \LaTeX{} templates}
\begin{abstract}
The manual for Bachelor thesis, Master thesis and Seminar \LaTeX{} templates at
Faculty of Electrical Engineering and Computing. The manual consists of the part which
explains the use of templates and the part which explains \LaTeX{} in general. The \LaTeX{}
part is made to be sufficient as an introduction to solving the text formatting
problems, but still relatively short.

\keywords{FER, \TeX{}.}
\end{abstract}

\end{document}
