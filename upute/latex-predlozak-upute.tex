% Upute za LaTeX predložak ZR-a i Seminara, verzija: 0.1alpha

\documentclass{zavrsni}
\usepackage{algorithmic}
\usepackage{algorithm}
\usepackage{listings}
\usepackage{longtable}
\usepackage{multicol}
\usepackage{url}
\lstset{language=Java, tabsize=4}

\begin{document}
\worknumber{000}
\title{Upute za korištenje \LaTeX~predloška za Završni rad i Seminar}
\author{Ivan Krišto \and Boran Car}

\maketitle

\tableofcontents

\chapter{Uvod}
% TODO: Čemu ovo
% TODO: Zašto je LaTeX moćan
% TODO: O organizaciji dokumenta

Nešto o moćnosti: \citep{taraborelli2009beauty}, \citep{cottrell1999word}.

\chapter{Upute za korištenje}
% TODO: Kratko i jasno za ljude koji znaju koristiti LaTeX
Osnova predloška je datoteka \texttt{zavrsni.cls}. Uz navedenu datoteku potrebna
je \texttt{.tex} datoteka za sadržaj rada te \texttt{.bib} datoteka za literaturu
(naziv \texttt{.bib} datoteke je proizvoljan, ali mora odgovarati argumentu koji
predate naredbi \verb|\bibliography{}| (vidi odjeljak \ref{sec:literatura})). Sve
navedene datoteke moraju se nalaziti u istom direktoriju.

Povezivanje \texttt{.tex} dokumenta i predloška iz \texttt{zavrsni.cls}
datoteke vrši se naredbom \verb|\documentclass{zavrsni}| koja mora biti prva
naredba u dokumentu.

Predložak, osim standardnih polja kao što \texttt{author} i \texttt{title}
sadrži obavezno polje \texttt{worknumber} --- broj Završnog rada. Osnovna
struktura \texttt{.tex} datoteke mora biti:
\begin{verbatim}
\documentclass{zavrsni}

% Ovdje možete staviti dodatne pakete koji
% su vam potrebni, npr.:
\usepackage{longtable}
\usepackage{multicol}

\begin{document}
\worknumber{000}
\title{Naslov rada}
\author{Vaše ime i prezime}
\maketitle
\tableofcontents
% Tu možete staviti popis slika i tablica

\chapter{Uvod}
% Sadržaj rada, ostala poglavlja i odjeljci.

\bibliography{literatura}
\bibliographystyle{plainnat}

% Dodatak nije obavezan
\appendix
\chapter{Poglavlje pod dodatkom}

\begin{sazetak}
Sažetak rada.

\vspace{15pt}
\noindent \textbf{Ključne riječi:} <popis ključnih riječi>
\end{sazetak}

\engtitle{Naslov rada na engleskom jeziku}
\begin{abstract}
Sažetak na engleskom jeziku.

\vspace{15pt}
\noindent \textbf{Keywords:} <popis ključnih riječi na engleskom>
\end{abstract}

\end{document}
\end{verbatim}

O stilu i strukturi dokumenta pogledajte poglavlje \ref{ch:stil-i-struktura}, a
za upoznavanje s \LaTeX--om te nekim njegovim posebnostima i receptima pogledajte
poglavlje \ref{ch:latex-upute}

Napravljeni predložak sadrži rješenje za dodavanje sažetka rada i posvete te
naredbu \texttt{engl}. Za navedeno pogledajte odjeljke \ref{sec:sazetak},
\ref{sec:posveta} i \ref{sec:engl}

Za pakete uključene u sam predložak pogledajte polja \texttt{RequirePackage} u
datoteci \texttt{završni.cls}.

\chapter{Stil i struktura}
\label{ch:stil-i-struktura}
% TODO: O stilu iz službenih MS Word uputa i općenito

\chapter{Upute za \LaTeX}
\label{ch:latex-upute}
U nastavku su dane upute za \LaTeX, no sužene za potrebe brzog uvoda u rad s
\LaTeX--om i konkretnim predloškom.

Ako se želite bolje upoznati s \LaTeX--om, konzultirajte poveznice navedene u
dodatku \ref{ch:korisne-stranice} te knjige poput \citep{oetiket2007lshort}.

\section{Instalacija \LaTeX~prevoditelja}

\subsection{Prevođenje \LaTeX~dokumenta}
% TODO: Napomena kako se zbog nekih stvari prevođenje mora pokrenuti više puta

\subsection{Dodavanje novih paketa (za prevoditelj)}


\section{Razvojna okruženja i pomoćni alati}


\section{Dodavanje popisa sadržaja, slika i tablica}
% TODO: Navesti kako se sadržaj gradi (svako poglavlje, odjeljak i sl. gradi
% sadržaj)

\section{Osnovno \LaTeX~formatiranje}
% TODO: emph, texttt, bold, crtice, razmaci i ``~'', navodnici, trotočka,
% prelazak u novi red, paragrafi i sl., komentari, znakovi koje treba escapeati


\section{Dodavanje fusnota i referenci}
Fusnote se dodaju naredbom \verb|\footnote{}|, npr.:
\begin{verbatim}
Popis fusnota\footnote{Objašnjenja koja se prikazuju
na dnu stranice} nije potreban.
\end{verbatim}
Rezultat navedenoga teksta je: ``Popis fusnota\footnote{Objašnjenja koja se
prikazuju na dnu stranice} nije potreban.''

Reference služe za povezivanje s nekim dijelom rada. Da bi ste omogućili
referenciranje na neki dio rada, tom dijelu morate postaviti oznaku
\engl{label} naredbom \verb|\label{}|, npr.:
\begin{verbatim}
\section{Dodavanje posvete}
\label{sec:posveta}
\end{verbatim}
a, referencirate se pomoću naredbe \verb|\ref{}|, npr.:
\begin{verbatim}
Za navedeno pogledajte odjeljke \ref{sec:sazetak},
\ref{sec:posveta} i \ref{sec:engl}
\end{verbatim}

Oznaku za referenciranje možete postaviti većini elemenata, a najčešće se
koristi za označavanje poglavlja, odjeljaka, jednadžbi, tablica i slika.


\section{Dodavanje listi, nabrajanja i definicija}
% TODO: Navesit kako je enumitem paket moćniji od ovog standardnog i sl.
Odličan prikaz mogućnosti \texttt{enumitem} paketa iznosi
\citep{collins2008enumitem}.


\section{Dodavanje tablica}


\section{Dodavanje slika}


\section{Dodavanje formula}


\section{Dodavanje programskog koda i sadržaja koji mora ostati neformatiran}


\section{Podjela sadržaja na više stupaca}


\section{Dodavanje literature}
\label{sec:literatura}
% TODO: neke stvari oko bibtexa, npr. {{sadržaj koji ima i npr. \LaTeX}}, ``_''
% u url-u i sl.
% Pogledati: http://en.wikibooks.org/wiki/LaTeX/Bibliography_Management

\subsection{Promjena načina citiranja}


\section{Dodavanje indeksa}


\section{Dodatci dokumenta \engl{appendix}}


\section{Dodavanje sažetka rada}
\label{sec:sazetak}
Sažetak rada je obavezan dio koji dolazi na sam kraj rada. Za sažetak na
Hrvatskom jeziku koristite okolinu \engl{environment} \emph{sazetak}, a za
sažetak na Engleskom jeziku okolinu \emph{abstract}. Prije korištenja
\emph{abstract} okoline potrebno je navesti naslov na engleskom jeziku naredbom
\verb|\engtitle{}|. Primjer korištenja:
\begin{verbatim}
\begin{sazetak}
Sažetak rada.

\vspace{15pt}
\noindent \textbf{Ključne riječi:} <popis ključnih riječi>
\end{sazetak}

\engtitle{Naslov rada na engleskom jeziku}
\begin{abstract}
Sažetak na engleskom jeziku.

\vspace{15pt}
\noindent \textbf{Keywords:} <popis ključnih riječi na engleskom>
\end{abstract}
\end{verbatim}
Nakon sažetka dolazi kraj dokumenta, odnosno naredba:
\begin{verbatim}
\end{document}
\end{verbatim}


\section{Dodavanje posvete}
\label{sec:posveta}
Posveta se dodaje naredbom \verb|\posveta{}| na mjestu gdje želite stranicu s
posvetom (npr.~prije tablice sadržaja --- iznad \verb|\tableofcontents|
naredbe). Primjer korištenja:
\begin{verbatim}
\posveta{Svima koji žele naučiti koristiti \LaTeX{}.}
\end{verbatim}


\section{Korištenje naredbe \texttt{engl}}
\label{sec:engl}
Naredba \verb|\engl{}| služi navođenju engleskog
prijevoda nekog termina (radi se o nestandardnoj naredbi), npr.:
\begin{verbatim}
Dodatak dokumenta \engl{appendix}.
\end{verbatim}
Rezultat je: ``Dodatak dokumenta \engl{appendix}.''


\section{Korištenje dodatnih paketa}
\LaTeX{} paketi se mogu promatrati kao dodatne biblioteke u programskim
jezicima. Dodavanje novih paketa vrši se \verb|\usepackage{}| naredbom.
Navedena naredba dolazi nakon \verb|\documentclass{}| naredbe. Primjer
korištenja:
\begin{verbatim}
\documentclass{zavrsni}
\usepackage{algorithmic}
\usepackage{algorithm}
\usepackage{listings}
\usepackage{longtable}
\usepackage{multicol}
\end{verbatim}

\LaTeX{} distribucije dolaze sa velikim brojem već instaliranih paketa.
Također, distribucije je moguće podesiti da, u slučaju korištenja paketa koji
nije instaliran, same dohvate paket sa CTAN--a i instaliraju ga.

Za više informacija o \LaTeX{} paketima, konzultirajte CTAN
(\url{http://tug.ctan.org/}). Korisne upute za korištenje i instalaciju \LaTeX{}
paketa možete naći na \url{http://en.wikibooks.org/wiki/LaTeX/Packages}.

Bitno je spomenuti da se uz svaki \TeX{} paket sa CTAN--a može skinuti pripadna
dokumentacija.


\section{Provjera pravopisa}
Programi \emph{aspell} i \emph{ispell} omogućavaju jednostavnu provjeru
pravopisa \LaTeX{} dokumenata (razumiju sintaksu \TeX{}--a). Primjer korištenja:
\begin{verbatim}
aspell -c dokument.tex
\end{verbatim}
ili
\begin{verbatim}
ispell dokument.tex
\end{verbatim}
Za više informacija, instalaciju i podešavanje navedenih alata konzultirajte:
\begin{itemize}
  \item \url{http://aspell.net/}
  \item \url{http://www.gnu.org/software/ispell/ispell.html}
  \item \url{http://cvs.linux.hr/spell/}
  \item \url{http://aspell.net/win32/}
  \item \url{http://gustav.fesb.hr/hr/ispell.html}
  \item \url{http://en.wikipedia.org/wiki/GNU_Aspell}
\end{itemize}


\bibliography{literatura}
\bibliographystyle{plainnat}


\appendix
\chapter{Korisne Internet stranice o \LaTeX--u}
\label{ch:korisne-stranice}
% TODO: Ovo nekako ljepše složiti
\textbf{LaTeX – A document preparation system:}
\begin{itemize}
  \item \url{http://www.latex-project.org/}
  \item Službena stranica \LaTeX~projekta.
\end{itemize}
\textbf{The Comprehensive TeX Archive Network (CTAN):}
\begin{itemize}
  \item \url{http://www.ctan.org/}
  \item Osnovna arhiva paketa i materijala vezanih uz \TeX~sustav.
\end{itemize}
\textbf{TeX Frequently Asked Questions on the Web:}
\begin{itemize}
  \item \url{http://www.tex.ac.uk/cgi-bin/texfaq2html}
  \item Veliki \TeX~i \LaTeX~FAQ sa konkretnim problemima i poveznicama na
  korisne stranice.
\end{itemize}
\textbf{The Not So Short Introduction to LATEX2e:}
\begin{itemize}
  \item \url{http://tobi.oetiker.ch/lshort/lshort.pdf}
  \item Izuzetno hvaljena knjiga o \LaTeX--u prigodna za početnike.
\end{itemize}
\textbf{Šime Ungar -- Ne baš tako kratak uvod u TeX s naglaskom na LaTeX2e:}
\begin{itemize}
  \item \url{http://web.math.hr/~ungar/lkratko2e_internet.pdf}
  \item Besplatna knjiga o \LaTeX--u na Hrvatskom jeziku.
\end{itemize}
\textbf{\LaTeX~wikibook:}
\begin{itemize}
  \item \url{http://en.wikibooks.org/wiki/LaTeX}
  \item Wiki stranica sa objašnjenjima i kvalitetnim \LaTeX~receptima.
\end{itemize}
\textbf{\LaTeX~courseware:}
\begin{itemize}
  \item \url{http://ahyco.ffri.hr/seminari2007/latex/home.html}
  \item Odličan \emph{courseware} pod nazivom \LaTeX~nastao u okviru kolegija
  ``Metodika nastave informatike II'' studijske grupe Matematika i informatika Filozofskog
  fakulteta u Rijeci.
\end{itemize}
\textbf{Detexify$^2$ -- \LaTeX~symbol classifier:}
\begin{itemize}
  \item \url{http://detexify.kirelabs.org/classify.html}
  \item Interaktivni prepoznavatelj simbola. Iznimno koristan alat ako tražite
  neki simbol.
\end{itemize}
\textbf{IEEE predavanje -- Uvod u \LaTeX:}
\begin{itemize}
  \item \url{http://www.fer.hr/ieee?\@=g4ct}
  \item Predavanje koje je u travnju 2007.~godine održao mr.~sc.~Tomislav
  Petković. Popratni materijali koji se mogu naći u repozitoriju su izuzetno
  korisni.
\end{itemize}
\textbf{\TeX{}ample TikZ and PGF:}
\begin{itemize}
  \item \url{http://www.texample.net/tikz/}
  \item Primjeri korištenja \texttt{tikz} paketa za izradu izuzetno složenih
  dijagrama.
\end{itemize}


% TODO: Napisati sažetak.
\begin{sazetak}
Sažetak rada.

\vspace{15pt}
\noindent \textbf{Ključne riječi:} Završni rad, Seminar, \LaTeX
\end{sazetak}

\engtitle{Manual for bachelor thesis and Seminar \LaTeX~templates}
\begin{abstract}
Sažetak na engleskom jeziku.

\vspace{15pt}
\noindent \textbf{Keywords:} bachelor thesis, seminar, \LaTeX
\end{abstract}

\end{document}
