\documentclass{beamer}

\mode<presentation>
{
\usetheme{default}
\setbeamercovered{invisible}
}
\usepackage[utf8]{inputenc}
\usepackage[croatian]{babel}
\usepackage[T1]{fontenc}
\usepackage{booktabs}
\usepackage{mathptmx}
\usepackage{palatino}
\usepackage{multicol}
\usepackage[absolute,overlay]{textpos}
\usepackage{listings}
\usepackage{booktabs}

\setbeamerfont{frametitle}{family=\rmfamily,size=\LARGE}
\setbeamerfont{title}{shape=\upshape,size=\Large,series=\bfseries}
\setbeamerfont{author}{series=\bfseries}
\setbeamerfont{date}{size=\small}
\setbeamerfont{institute}{size=\small}
\definecolor{mso2007}{RGB}{0,124,168}
\setbeamercolor{frametitle}{fg=mso2007}
\setbeamercolor{title}{fg=mso2007}
\setbeamercolor{item}{fg=mso2007}
\setbeamercolor{section in toc}{fg=black,parent=structure,series=\bfseries}
\setbeamertemplate{sections/subsections in toc}[circle]
\setbeamertemplate{sidebar right}{}% or get rid of navigation entries there somehow
\setbeamertemplate{footline}[frame number]
\setbeamercolor{footline}{fg=white}

\setbeamertemplate{frametitle}
{
\insertframetitle
\vspace{15pt} % Nakon ovoga mora ići prazna linija!!

}

\title{Pisanje znanstvenih radova u \LaTeX-u}
\author{Ivan Krišto}

\institute[Universities of]
{
Fakultet Elektrotehnike i Računarstva\\
Sveučilište u Zagrebu}

\date{08.03.2011.}

\begin{document}

\begin{frame}
\begin{textblock}{7}(1,1.2)
\begin{figure}
\centering
\includegraphics[width=1.5cm]{logo-sveuciliste.jpg}
\end{figure}
\end{textblock}

\begin{textblock}{7}(7.2,1.2)
\begin{figure}
\centering
\includegraphics[width=1.25cm]{FER_logo.jpg}
\end{figure}
\end{textblock}

\begin{textblock}{7}(12,1.6)
\begin{figure}
\centering
\includegraphics[width=2.5cm]{logo-ieee.png}
\end{figure}
\end{textblock}

\vspace{70pt}
\titlepage
\end{frame}

\usebackgroundtemplate{\includegraphics[width=\paperwidth]{background}}

\begin{frame}[t]{Organizacija}
\textbf{Cilj:} upoznavanje s \LaTeX-om, njegovim prednostima u odnosu na
tekstualne procesore, uz konkretnu primjenu na pisanje znanstvenih radova.

\begin{multicols}{2}
\tableofcontents
\end{multicols}
\end{frame}

% Done
% TODO provjeri još jednom
\section{Što je \LaTeX{}}
\begin{frame}[t]{Što je \LaTeX{}}
\begin{itemize}
  \item Sustav za pripremu dokumenata i \emph{slovoslagarstvo} (engl.\ \emph{typesetting})
  \item Nadogradnja \TeX-a
  \item Usmjeravanje posla na strukturu i sadržaj (ne izgled)
  \item Jednostavna izrada lijepih dokumenata po predefiniranim stilovima
  \item Veliki broj dodataka (izrada tablica, grafova, crteža, \ldots)
  \item Veći stilski zahvati znaju biti izuzetno teški
  \item Nije zamišljen kao WYSIWYG alat
\end{itemize}
\end{frame}

\subsection{Kako \LaTeX{} radi}
\begin{frame}[t]{Kako \LaTeX{} radi}
\begin{figure}[htb]
\begin{center}
\begin{picture}(270,15)
\put(13,5){\makebox(40,15){\texttt{dokument.tex}}}
\put(74,12){\vector(1,0){15}}
\put(91,5){\framebox(90,15){\texttt{pdflatex}}}
\put(183,12){\vector(1,0){15}}
\put(219,4){\makebox(40,15){\texttt{dokument.pdf}}}
\end{picture}
\caption{Tok \LaTeX\ prevođenja}
\label{fig:ru-sustav-dijagram}
\end{center}
\end{figure}
\begin{itemize}
  \item \texttt{dokument.tex} -- tekstualni dokument sa sadržajem i \LaTeX\ naredbama
  \item \texttt{pdflatex} -- prevođenje \texttt{tex} dokumenta u \texttt{pdf}\\
  (prevoditelj \texttt{latex} prevodi u \texttt{dvi})
  \pause
  \item Distribucije:
  \begin{itemize}
    \item Windowsi -- Mik\TeX{} (\url{http://miktex.org/})
    \item Linux -- \TeX\ Live (\url{http://www.tug.org/texlive/}; dostupan također za Windowse i MacOS)
    \item Sadrže više prevoditelja i skup korisnih alata (\texttt{ps2pdf}, \texttt{ebb}, \ldots)
  \end{itemize}
\end{itemize}
\end{frame}

\section{Zašto koristiti \LaTeX}
\begin{frame}[t]{Zašto koristiti \LaTeX\ (1)}
\begin{itemize}
  \item Odjeljivanje sadržaja i strukture od stila -- jednostavna promjena predložaka (primjerice, kod promjene konferencije)
  \item Dokumenti su prenosivi kroz verzije \LaTeX\ distribucija (primjerice, dokument napisan MS Wordom 2007 nije čitljiv u MS Wordu 2000)
  \item Izuzetno dobro vođenje referenci -- {\sc Bib}\TeX{}
  \item Mogućnost korištenja sustava za verzioniranje (VCS sustavi -- SVN, Mercurial, Git, \ldots) -- izuzetno korisno kod tekstova sa više autora!
  \item Mogućnost umetanja komentara pri čemu se ne narušava dizajn
\end{itemize}
\end{frame}

\begin{frame}[t]{Zašto koristiti \LaTeX\ (2)}
\begin{itemize}
  \item Dokumenti se pišu kao obične tekstualne datoteke s \TeX\ sintaksom --
  različite verzije teksta su lako usporedive \emph{diff} alatima % note: trend je odmicanje od binarnog zapisa -- odf i docx su zapravo XML
  \item Mogućnost pisanja dokumenata u dijelovima (naredba \texttt{include})
  \item Trudi se poštivati osnove tipografije -- poštivanje tipografskih pravila poboljšava čitljivost  
\end{itemize}
\end{frame}

\section{Razvojna okruženja}
\begin{frame}[t]{Razvojna okruženja}
\begin{itemize}
\item TexMaker (\url{http://www.xm1math.net/texmaker/})
\begin{description}
  \item[+] Veliki broj gotovih primjera %(odlično za početnike!)
  \item[--] Pretrpano sučelje
\end{description}
\pause
\item \TeX{}works (\url{http://code.google.com/p/texworks/})
\begin{description}
  \item[+] Jednostavan i malen
  \item[+] Ugrađeni preglednik
  \item[--] Nema gotovih primjera (simbola i konstrukcija)
\end{description}
\pause
\item Eclipse \& \TeX{}lipse (\url{http://www.eclipse.org/} \& \url{http://texlipse.sourceforge.net/})
\begin{description}
  \item[+] Najmoćnije razvojno okruženje
  \item[+] Odličan prikaz grešaka (u razini linije!)
  \item[+] Autocomplete čita čak i \texttt{bib} datoteke
  \item[+] Prevodi paralelno s radom
  \item[--] Glomazan
\end{description} 
\end{itemize}
\end{frame}

\section{Osnove \LaTeX-a}
\subsection{\LaTeX{} dokument}
\begin{frame}[fragile]
\frametitle{Struktura \LaTeX{} dokumenta}
\begin{multicols}{2}
Osnovni \LaTeX\ dokument:
\begin{verbatim}
\documentclass{article}
\title{Naslov}
\author{Autor}
\begin{document}
\section{Uvod}
Uvod..
\subsection{Unutar uvoda}
Prvi paragraf.

Drugi paragraf.
\end{document}
\end{verbatim}

\begin{itemize}
  \item Stil se definira \verb|\documentclass| naredbom
  \item Sadržaj se nalazi u \texttt{document} okolini
  \item Struktura se gradi \verb|\section|, \verb|\subsection| i
  \verb|\subsubsection| naredbama te podjelom na paragrafe
  \item Tekst se odjeljuje u paragrafe praznim redom
  \item Prelazak u novi red (bez stvaranja paragrafa) -- sekvenca: \verb|\\|
\end{itemize}
\end{multicols}
\end{frame}

\subsection{Naredbe i okoline}
\begin{frame}[fragile]
\frametitle{Naredbe i okoline}
\begin{description}
  \item[Naredbe:] Elementi \LaTeX-a koji primaju argumente i parametre te se transformiraju u
  oblikovani tekst
  \begin{itemize}
    \item Sintaksa: \verb|\naziv[parametar][...]{argument}{...}|
    \item Primjeri: \verb|\tableofcontents|, \verb|\emph{fakultet}|, \verb|\frac{2}{x+2}|, \verb|\usepackage[croatian]{babel}|
  \end{itemize}
  \item[Okoline:] Slično kao naredbe, ali umjesto argumenta primaju
  segment dokumenta (razlika je u obradi paragrafa)
  \begin{itemize}
    \item Sintaksa: \verb|\begin{naziv}[parametar][..]{argument}{..}| \ldots \verb|\end{naziv}|
    \item Primjer: \verb|\begin{tabular}{ l c r }| \ldots \verb|\end{tabular}|
  \end{itemize}
\end{description}
\end{frame}

% \subsection{Liste, nabrajanja i opisi}
% \begin{frame}[fragile]
% \frametitle{Liste, nabrajanja i opisi}
% \begin{multicols}{2}
% \begin{itemize}
%   \item Liste:
% \begin{verbatim}
% \begin{itemize}
%   \item kruh
%   \item mlijeko
%   \item jaja
% \end{itemize}
% \end{verbatim}
%   \item Nabrajanja:
% \begin{verbatim}
% \begin{enumerate}
%   \item Prva
%   \item Druga
%   \item Treća
% \end{enumerate}
% \end{verbatim}
%   \item Opisi:
% \begin{verbatim}
% \begin{description}
%   \item[Pingvin] ptica
%   \item[Špar] riba
% \end{description}
% \end{verbatim}
%   \item Gniježđenje:
% \begin{verbatim}
% \begin{itemize}
%   \item Prva stavka
%   \begin{enumerate}
%     \item Numerirano
%   \end{enumerate}
% \end{itemize}
% \end{verbatim}
% \end{itemize}
% \end{multicols}
% \end{frame}

\subsection{Slike i tablice}
\begin{frame}[fragile]
\frametitle{Dodavanje slika}
\begin{itemize}
  \item Dodaju se pomoću \texttt{figure} lebdeće okoline
  (slika se ne mora pojaviti tamo gdje je navedete!)
\begin{verbatim}
\begin{figure}[htb]
\centering
\includegraphics[width=2cm]
                {FER_logo.jpg}
\caption{Logo FER-a}
\label{fig:fer-logo}
\end{figure}
\end{verbatim}
\end{itemize}
\begin{textblock}{10}(8,5.3)
\begin{figure}
\centering
\includegraphics[width=2cm]{FER_logo.jpg}
\caption{Logo FER-a}
\label{fig:fer-logo}
\end{figure}
\end{textblock}
\begin{itemize}
  \item \texttt{htb} -- pozicioniranje (\emph{here}, \emph{top}, \emph{bottom});
  \texttt{caption} -- naslov (uvijek na dnu); \texttt{label} -- za referenciranje pomoću {\tt ref} naredbe.
\end{itemize}
\end{frame}

\begin{frame}[fragile]
\frametitle{Dodavanje tablica (1)}
\begin{itemize}
  \item Ostvaruju se preko \texttt{table} i \texttt{tabular} okolina
  \item \texttt{table} je lebdeća okolina unutar koje se
  smješta \texttt{tabular} okolina s podatcima
  \item Znak \& odvaja stupce; sekvenca \textbackslash\textbackslash\ odvaja retke 
\begin{verbatim}
\begin{table}[htb]
\caption{Konstante}
\centering
\begin{tabular}{llr} \hline
Konstanta & Opis & Vrijednost \\ \hline
$\pi$ & Pi & 3.14159 \\
$e$ & Eulerov broj & 2.71828 \\
$\varphi$ & Zlatni rez & 1.61803 \\ \hline
\end{tabular}
\end{table}
\end{verbatim}
\end{itemize}
\end{frame}

\begin{frame}[fragile]
\frametitle{Dodavanje tablica (2)}
\begin{table}[htb]
\caption{Konstante}
\centering
\begin{tabular}{llr} \hline
Konstanta & Opis & Vrijednost\\ \hline
$\pi$ & Pi & 3.14159 \\
$e$ & Eulerov broj & 2.71828 \\
$\varphi$ & Zlatni rez & 1.61803 \\ \hline
\end{tabular}
\end{table}
% \begin{table}[htb]
% \caption{Konstante (\texttt{booktabs} paket)}
% \centering
% \begin{tabular}{llr} \toprule
% Konstanta & Opis & Vrijednost\\ \midrule
% $\pi$ & Pi & 3.14159 \\
% $e$ & Eulerov broj & 2.71828 \\
% $\varphi$ & Zlatni rez & 1.61803 \\ \bottomrule
% \end{tabular}
% \end{table}
\end{frame}

\subsection{Matematički izrazi}
\begin{frame}[fragile]
\frametitle{Matematički izrazi}
\begin{itemize}
  \item Brzo i jednostavno (uz malo vježbanja)
  \item Piše se u modu za matematiku
  \begin{itemize}
    \item \verb|$ ... $| za izraze u istoj liniji uz tekst
    \item \verb|\[  ... \]| (ili \verb|$$  ... $$|) za izraze u posebnoj liniji
    \item okolina {\tt equation} -- dodaje brojač
  \end{itemize}
  \item Primjeri: \verb|$x^2$| -- $x^2$, \verb|$\int_{2}^{10} x^2 dx$| -- $\int_{2}^{10} x^2 dx$
  \[  f(t)\ast g(t) = \int^{\infty}_{-\infty} f(\tau)g(t-\tau)d\tau \]
\begin{verbatim}
\[   f(t)\ast g(t) = \int^{\infty}_{-\infty}
                     f(\tau)g(t-\tau)d\tau   \]
\end{verbatim}
  \item Dobre upute za pisanje matematičkih izraza u \LaTeX-u:
  \url{ftp.ams.org/pub/tex/doc/amsmath/short-math-guide.pdf}
\end{itemize}
\end{frame}

% \subsection{Neformatirani sadržaj}
% \lstset{language=C, tabsize=2}
% \begin{frame}[fragile]
% \frametitle{Neformatirani sadržaj}
% \begin{multicols}{2}
% \begin{itemize}
%   \item Neformatirani sadržaj -- tekst koji se mora prikazati s očuvanim izvornim razmacima, redovima i neekspandiranim naredbama
%   \item Linijski, naredba \verb|\verb|, primjer: ``\verb@\verb|\emph{nešto}|@''
%   \item Višelinijski, okolina \texttt{verbatim}:
%   \begin{verbatim}
%   Linija prva,
%     druga
%       i treća.
%   \end{verbatim}
%   \item Programski kod se može uključiti unutar \texttt{verbatim} okoline
%   \item Posebni paketi za uključivanje programskog koda, primjerice \texttt{listings}:
% \end{itemize}
%   \begin{lstlisting}
%   int fib(int n) {
%    if (n <= 0) return 0;
%    else if(n==1)return 1;
%    else return fib(n-1)
%   	          + fib(n-2);
%   }
%   \end{lstlisting}
% \end{multicols}
% \end{frame}

\subsection{Literatura}
\begin{frame}[fragile]
\frametitle{Literatura}
\begin{itemize}
  \item Upravljanjem literature bavi se {\sc Bib}\TeX
  \item Literatura se navodi u \texttt{bib} datoteci ({\sc Bib}\TeX\ zapisi)
\begin{verbatim}
@article{greenwade93,
 author = {Greenwade, D., George},
 title = {The Comprehensive Tex Archive Network},
 year = {1993},
 journal = {TUGBoat}
}
\end{verbatim}
\item Veza \texttt{literatura.bib} i \texttt{tex} datoteka: \verb|\bibliography{literatura}|
\item Postavljanje stila popisa literature \verb|\bibliographystyle{plain}|% (stil ``\texttt{plain}'')
\item Referenciranje na literaturu: \verb|\cite{greenwade93}|
\item Prevođenje: \texttt{pdflatex}, \texttt{bibtex}, \texttt{pdflatex}, \texttt{pdflatex}
\item \texttt{natbib} paket -- sortiranje, stil, referenciranje preko \verb|\citep|, \verb|\citet| 
\end{itemize}
\end{frame}

\subsection{Paketi}
\begin{frame}[t]{Paketi}
\begin{itemize}
  \item Skupovi definicija naredbi i okolina
  \item Glavni izvor: ``The Comprehensive \TeX\ Archive Network (CTAN)''
  \item Ručna instalacija je \emph{relativno} složena
  \item \LaTeX\ distribucije uglavnom same skinu i instaliraju paket koji zatražite
  \item Dodaju se \texttt{usepackage} naredbom odmah nakon \texttt{documentclass} naredbe
  \item Par korisnih paketa:
  \begin{description}
	\item[natbib] olakšavanje rada s literaturom
	\item[booktabs] naredbe za stiliziranje tablica
	\item[enumitem] proširenje mogućnosti listi i nabrajanja
	\item[amsmath] proširenje mogućnosti za pisanje matematičkih izraza
	\item[multicol] naredbe za pisanje u stupcima
  \end{description}
\end{itemize}
\end{frame}

\subsection{Specifičnosti i savjeti}
\begin{frame}[fragile]
\frametitle{Specifičnosti i savjeti}
\begin{itemize}
  \item<1-> Posebni znakovi (osnova \TeX\ sintakse): \$ \% \_ \} \{ \& \# \textbackslash
  \begin{itemize}
    \item Koristi se isključivanje (engl.\ \emph{escape}), primjerice \verb|\$|, a \textbackslash\ se dobiva naredbom \verb|\textbackslash|
  \end{itemize}
  \item<2-> Može se dogoditi da je riječ loše prelomljena te prelazi rub stranice
  \begin{itemize}
    \item Razlomite riječ: \verb|pre\-lo\-mlje\-na| ili \verb|ra\-ču\-na\-lo|
  \end{itemize}
  \item<3-> Pravopis možete provjeriti pomoću \emph{aspell} ili \emph{ispell} alata, primjerice: \texttt{aspell -c dokument.tex}
  \item<4-> Ponekad je potrebno prevođenje pokrenuti više puta (primjerice, kod dodavanja, uklanjanja ili mijenjanja naziva odjeljaka)
  \item<5-> Acrobat Reader (MS Windows) ``zaključa'' {\tt pdf} datoteku -- prevođenjem {\tt
  tex} datoteke uz otvoreni Acrobat Reader rezultira greškom (možete koristiti
  \emph{Evince})
\end{itemize}
\end{frame}

\section{Korištenje predložaka za znanstvene radove}
\begin{frame}[fragile]
\frametitle{O predlošcima za radove}
\begin{itemize}
  \item Predlošci se obično nalaze u sklopu dijela ``Instructions for Authors'' -- \emph{obavezno} ih slijedite!
  \item Uz predloške uvijek dolaze opširne upute
  \item Postoje predlošci za sadržaj ({\tt cls}) i navode literature ({\tt bst})
  \item Instalacija: kopirajte ih u direktorij s {\tt tex} datotekom
  \item Često korišteni predlošci: Springer (LNCS), IEEE, ACM
  \item Obratite pozornost na opcije stila (opcije {\tt documentclass} naredbe)
  i novodefinirane naredbe (primjerice, {\tt keywords}, {\tt email}, i sl.)
  \item Ako postoji pokazni primjer predloška, iskoristite ga
  \item Uz imena autora morate navesti i institucije (engl.\ \emph{affiliations})
\end{itemize}
\end{frame}

\begin{frame}[fragile]
\frametitle{Springer}
\begin{itemize}
  \item Predložak: \url{ftp://ftp.springer.de/pub/tex/latex/llncs/latex2e/llncs2e.zip}
  \item Osnova predloška: \texttt{llncs.cls} \& \texttt{splncs03.bst} -- definira \texttt{llncs} klasu
  \item Povezivanje predloška s {\tt tex} datotekom: \scriptsize \verb|\documentclass{llncs}| \& \verb|\bibliographystyle{splncs03}| \normalsize
\end{itemize}
\vspace{-10pt}
\scriptsize
\begin{verbatim}
     \title{Naslov}
     \author{Ivan Krišto \and Groucho Marx}
     \institute{Faculty of Electrical and Computing Engineering\\
     University of Zagreb\\
     Unska 3, 10000 Zagreb, Croatia\\
     \email{\{ivan.kristo, groucho.marx\}@fer.hr}}

     \maketitle

     \begin{abstract}
     Sažetak.

     \textbf{Keywords:} primjer, \LaTeX{}, Springer.
     \end{abstract}
\end{verbatim}
\normalsize
\end{frame}

\begin{frame}[fragile]
\frametitle{IEEE}
\begin{itemize}
  \item Predložak: \small \url{www.ieee.org/documents/IEEEtran5.zip}, \normalsize 
  literatura: \small \url{www.ieee.org/documents/IEEEtranBST2.zip} \normalsize 
  \item Osnova predloška: \small \texttt{IEEEtran.cls} \& \texttt{IEEEtran.bst} \normalsize (više
  verzija, preporuka je \small \texttt{IEEEtranS.bst} \normalsize ili \small \texttt{IEEEtranNS.bst}) \normalsize
  \item Povezivanje predloška s {\tt tex} \& {\tt bib} datotekama:
  \scriptsize
  \begin{verbatim}
\documentclass[conference]{IEEEtran}
...  
\bibliographystyle{IEEEtran}
\bibliography{IEEEabrv,literatura}
  \end{verbatim}
  \normalsize
  \item Dodavanje autora:
\scriptsize
\begin{verbatim}
\author{\IEEEauthorblockN{Groucho Marx}
\IEEEauthorblockA{School of Good Entertainment\\
Holywood\\
Los Angeles, California, USA\\
Email: president@freedonia.gov}
\and
\IEEEauthorblockN{Homer Simpson}
\IEEEauthorblockA{Twentieth Century Fox\\
Springfield, USA\\
Email: homer@thesimpsons.com}}
\end{verbatim}
\normalsize
\end{itemize}
\end{frame}

\begin{frame}[fragile]
\frametitle{IEEE -- napomene}
\begin{itemize}
  \item Uz predloške dolaze sažete upute i primjeri
  \item Primjeri sadrže dosta korisnih informacija (zakomentirani unutar {\tt tex} datoteke)
  \item Preporuka: koristite \texttt{IEEEtranNS.bst} za stil literature uz {\tt natbib} paket
\end{itemize}
\end{frame}

\begin{frame}[fragile]
\frametitle{ACM (1)}
\begin{itemize}
  \item Predložak: \small \url{http://www.acm.org/sigs/publications/acm_proc_article-sp.cls}\normalsize, za
  literaturu koristi se standardni {\tt abbrv} {\sc Bib}\TeX\ stil.
  \item Povezivanje predloška s {\tt tex} \& {\tt bib} datotekama:
  \scriptsize
  \begin{verbatim}
\documentclass{acm_proc_article-sp}
...  
\bibliographystyle{abbrv}
\bibliography{literatura}
  \end{verbatim}
  \normalsize
\end{itemize}
\end{frame}

\begin{frame}[fragile]
\frametitle{ACM (2)}
\begin{itemize}
  \item Dodavanje autora:
\scriptsize
\begin{verbatim}
\numberofauthors{2}
\author{
\alignauthor
Ivan Krišto\\
       \affaddr{Faculty of Electrical and Computing Engineering}\\
       \affaddr{University of Zagreb}\\
       \affaddr{Unska 3, 10000 Zagreb, Croatia}\\
       \email{ivan.kristo@fer.hr}
\and
\alignauthor
Groucho Marx\\
       \affaddr{School of Good Entertainment}\\
       \affaddr{Holywood}\\
       \affaddr{Los Angeles, California, USA}\\
       \email{president@freedonia.gov}
}
\end{verbatim}
\normalsize
\end{itemize}
\end{frame}

\subsection{Specifičnosti i savjeti}
\begin{frame}[fragile]
\frametitle{Specifičnosti i savjeti (1)}
\begin{itemize}
  \item<1-> Ako možete koristiti \texttt{natbib}, koristite ga
  \item<2-> Kodnu stranicu uvijek postavite na UTF-8:
  \begin{verbatim}
\usepackage[utf8]{inputenc}
\usepackage[english]{babel}
\usepackage[T1]{fontenc}
  \end{verbatim}
  \item<3-> Odjeljak ``Acknowledgement'' se ne numerira, koristite okolinu {\tt section*} (primjeti $\star$)
  \item<4-> Ako imate dvostupčani predložak te veliku tablicu ili sliku, koristite okolinu {\tt table*} odnosno {\tt figure*}
  \item<5-> Matematičke izraze uvijek numerirajte:
  \begin{verbatim}
\begin{equation}
  \sin^2\varphi + \cos^2\varphi = 1
  \label{equ:trigonometry-first-lady}
\end{equation}
  \end{verbatim}
\end{itemize}
\end{frame}

\begin{frame}[fragile]
\frametitle{Specifičnosti i savjeti (2)}
\begin{itemize}
  \item Kod dvostupčanih predložaka često se zahtjeva balansiranje zadnje stranice --- uključite: \verb|\usepackage{flushend}|
  \item {\sc Bib}\TeX\ citate nude: \url{http://scholar.google.com}, \url{http://citeseerx.ist.psu.edu},
  \url{http://www.springerlink.com}, \url{http://www.sciencedirect.com}, \url{http://portal.acm.org}, \ldots.
  \item Ako koristite javni VCS repozitorij (primjerice \url{code.google.com}), pazite na licence!
  \item \emph{Diff} alati uz VCS su izuzetno korisni! Primjeri: \emph{Meld} (Linux), \emph{WinMerge} (Windows)
\end{itemize}
\end{frame}

\begin{frame}
\frametitle{Pitanja?}
Korisna literatura:
\begin{itemize}
  \item \LaTeX\ wikibook -- wiki s puno primjera: \url{http://en.wikibooks.org/wiki/LaTeX}
  \item Pitanja i odgovori (programiranje \& \TeX): \url{http://stackoverflow.com/} \& \url{http://tex.stackexchange.com/}
  \item ``Ne baš tako kratak uvod u \TeX\ s naglaskom na \LaTeX:'' \url{web.math.hr/~ungar/lkratko2e_internet.pdf}
  \item ``\TeX\ Frequently Asked Questions on the Web:'' \url{http://www.tex.ac.uk/cgi-bin/texfaq2html}
  \item Upute za pisanje radova na FER-u korištenjem \LaTeX-a: \url{http://fer-latex-templates.googlecode.com/files/latex-predlozak-upute.pdf}
\end{itemize}

\end{frame}

\end{document}
