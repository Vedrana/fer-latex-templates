% Upute za LaTeX predložak ZR-a i Seminara, verzija: 0.1alpha

\documentclass{zavrsni}
\usepackage{algorithmic}
\usepackage{algorithm}
\usepackage{listings}
\usepackage{longtable}
\usepackage{multicol}
\lstset{language=Java, tabsize=4}

\begin{document}
\worknumber{000}
\title{Upute za korištenje \LaTeX~predloška za Završni rad i Seminar}
\author{Ivan Krišto \and Boran Car}

\maketitle

\tableofcontents

\chapter{Uvod}
% TODO: Čemu ovo
% TODO: Zašto je LaTeX moćan
% TODO: O organizaciji dokumenta

\chapter{Upute za korištenje}
% TODO: Kratko i jasno za ljude koji znaju koristiti LaTeX
Osnova predloška je datoteka \texttt{zavrsni.cls}. Uz navedenu datoteku potrebna
je \texttt{.tex} datoteka za sadržaj rada te \texttt{.bib} datoteka za literaturu
(naziv \texttt{.bib} datoteke je proizvoljan, ali mora odgovarati argumentu koji
predate naredbi \verb|\bibliography{}| (vidi odjeljak \ref{sec:literatura})). Sve
navedene datoteke moraju se nalaziti u istom direktoriju.

Povezivanje \texttt{.tex} dokumenta i predloška iz \texttt{zavrsni.cls}
datoteke vrši se naredbom \verb|\documentclass{zavrsni}| koja mora biti prva
naredba u dokumentu.

Predložak, osim standardnih polja kao što \texttt{author} i \texttt{title}
sadrži obavezno polje \texttt{worknumber} --- broj Završnog rada. Osnovna
struktura \texttt{.tex} datoteke mora biti:
\begin{verbatim}
\documentclass{zavrsni}

% Ovdje možete staviti dodatne pakete koji
% su vam potrebni, npr.:
\usepackage{longtable}
\usepackage{multicol}

\begin{document}
\worknumber{000}
\title{Naslov rada}
\author{Vaše ime i prezime}
\maketitle
\tableofcontents
% Tu možete staviti popis slika i tablica

\chapter{Uvod}
% Sadržaj rada, ostala poglavlja i odjeljci.

\bibliography{literatura}
\bibliographystyle{plainnat}

% Dodatak nije obavezan
\appendix
\chapter{Poglavlje pod dodatkom}

\begin{sazetak}
Sažetak rada.

\vspace{15pt}
\noindent \textbf{Ključne riječi:} <popis ključnih riječi>
\end{sazetak}

\engtitle{Naslov rada na engleskom jeziku}
\begin{abstract}
Sažetak na engleskom jeziku.

\vspace{15pt}
\noindent \textbf{Keywords:} <popis ključnih riječi na engleskom>
\end{abstract}

\end{document}
\end{verbatim}

O stilu i strukturi dokumenta pogledajte poglavlje \ref{ch:stil-i-struktura}, a
za upoznavanje s \LaTeX--om te nekim njegovim posebnostima i receptima pogledajte
poglavlje \ref{ch:latex-upute}

Napravljeni predložak sadrži rješenje za dodavanje sažetka rada i posvete te
naredbu \texttt{engl}. Za navedeno pogledajte odjeljke \ref{sec:sazetak},
\ref{sec:posveta} i \ref{sec:engl}

Za pakete uključene u sam predložak pogledajte polja \texttt{RequirePackage} u
datoteci \texttt{završni.cls}.

\chapter{Stil i struktura}
\label{ch:stil-i-struktura}
% TODO: O stilu iz službenih MS Word uputa i općenito

\chapter{Upute za \LaTeX}
\label{ch:latex-upute}
% TODO: O LaTeX-u

\section{Instalacija \LaTeX~prevoditelja}

\subsection{Prevođenje \LaTeX~dokumenta}
% TODO: Napomena kako se zbog nekih stvari prevođenje mora pokrenuti više puta

\subsection{Dodavanje novih paketa (za prevoditelj)}

\section{Razvojna okruženja i pomoćni alati}

\section{Dodavanje popisa sadržaja, slika i tablica}

\section{Osnovno \LaTeX~formatiranje}
% TODO: emph, texttt, bold, crtice, razmaci i ``~'', navodnici, trotočka,
% prelazak u novi red, paragrafi i sl., 

\section{Dodavanje listi, nabrajanja i definicija}

\section{Dodavanje tablica}

\section{Dodavanje slika}

\section{Dodavanje formula}

\section{Dodavanje programskog koda i sadržaja koji mora ostati neformatiran}

\section{Dodavanje fusnota i referenci}

\section{Podjela sadržaja na više stupaca}

\section{Dodavanje literature}
\label{sec:literatura}

\subsection{Promjena načina citiranja}

\section{Dodavanje indeksa}

\section{Dodatci dokumenta \engl{appendix}}

\section{Dodavanje sažetka rada}
\label{sec:sazetak}

\section{Dodavanje posvete}
\label{sec:posveta}

\section{Korištenje naredbe \texttt{engl}}
\label{sec:engl}

\section{Dodavanje novih paketa (za korištenje)}

\bibliography{literatura}
\bibliographystyle{plainnat}

\appendix
\chapter{Statistika rezultata čišćenja}
Neki dodatak.

\begin{sazetak}
Za razliku od tradicionalnih tekstovnih dokumenata, web stranice tipično
sadržavaju veliku količinu informacija koje se ne odnose izravno na njihov
sadržaj, poput promidžbenih poruka, navigacijskih uputa, i sl. U kontekstu
dubinske analize teksta i računalno-lingvističke obrade, takve informacije
predstavljaju neželjeni šum.

U okviru rada proučeni su postupci za automatsko čišćenje dokumenata u HTML-u od
nepotrebnog sadržaja, razvijena programska implementacija postupka pogodna za
ugradnju u pobirač dokumenata s web sjedišta te provodeno eksperimentalno
vrednovanje postupka.

\vspace{15pt}
\noindent \textbf{Ključne riječi:} HTML, web stranice, uklanjanje šuma,
automatsko čišćenje, dubinska analiza teksta
\end{sazetak}

\engtitle{Web page cleaning techniques for text mining}
\begin{abstract}
Unlike traditional text documents, web pages typically contain large
amount of information that doesn't refer to content of web page
directly, for example advertisements, navigation etc. In context of
text mining and computational linguistic processing, such information
represent unwanted noise.

This work describes automated web page cleaning techniques and
presents program implementation and experimental evaluation of a
cleaning technique suitable for using with web crawler.

\vspace{15pt}
\noindent \textbf{Keywords:} HTML, web pages, boilerplate removal, automated web
cleaner, text mining
\end{abstract}

\end{document}
