% Upute za LaTeX predložak ZR-a i Seminara, verzija: 0.1pre-alpha

\documentclass{zavrsni}
\usepackage{algorithmic}
\usepackage{algorithm}
\usepackage{listings}
\usepackage{longtable}
\usepackage{multicol}
\lstset{language=Java, tabsize=2}

\begin{document}
\thesisnumber{000}
\title{Upute za korištenje \LaTeX{} predloška za Završni rad i Seminar}
\author{Ivan Krišto \and Boran Car}

\maketitle

\tableofcontents

\chapter{Uvod}
% TODO: Čemu ovo
% TODO: Zašto je LaTeX moćan
% TODO: Navesti da je latex moćan, da su standardne stvari jednostavne i
% lijepe, ali da za one koje mogu napraviti kaos se treba potruditi :D
% TODO: O organizaciji dokumenta

% Navesti kako oni koji već znaju koristiti latex trebaju samo pročitati
% \ref{ch:upute} i \ref{ch:stil-i-struktura}, dok je ostatak dokumenta
% namjenjen osobama koje se prvi put susreću s LaTeX--om.

Iskoristiti: \citep{taraborelli2009beauty}, \citep{cottrell1999word}.

\chapter{Upute za korištenje}
\label{ch:upute}
% Kratko i jasno za ljude koji znaju koristiti LaTeX.
Osnova predloška je datoteka \texttt{zavrsni.cls}. Uz navedenu datoteku potrebna
je \texttt{tex} datoteka za sadržaj rada te \texttt{bib} datoteka za literaturu
(naziv \texttt{bib} datoteke je proizvoljan, ali mora odgovarati argumentu koji
predate naredbi \verb|\bibliography{}| (vidi odjeljak \ref{sec:literatura})). Sve
navedene datoteke moraju se nalaziti u istom direktoriju.

Povezivanje \texttt{tex} dokumenta i predloška iz \texttt{zavrsni.cls}
datoteke vrši se naredbom \verb|\documentclass{zavrsni}| koja mora biti prva
naredba u dokumentu.

Predložak, osim standardnih polja kao što \texttt{author} i \texttt{title}
sadrži obavezno polje \texttt{thesisnumber} --- broj Završnog rada. Osnovna
struktura \texttt{tex} datoteke mora biti:
\begin{verbatim}
\documentclass{zavrsni}

% Ovdje možete staviti dodatne pakete koji
% su vam potrebni, npr.:
\usepackage{longtable}
\usepackage{multicol}

\begin{document}
\thesisnumber{000}
\title{Naslov rada}
\author{Vaše ime i prezime}
\maketitle
\tableofcontents
% Tu možete staviti popis slika i tablica

\chapter{Uvod}
% Sadržaj rada, ostala poglavlja i odjeljci.

\bibliography{literatura}
\bibliographystyle{plainnat}

% Dodatak nije obavezan
\appendix
\chapter{Poglavlje pod dodatkom}

\begin{sazetak}
Sažetak rada.

\vspace{15pt}
\noindent \textbf{Ključne riječi:} <popis ključnih riječi>
\end{sazetak}

\engtitle{Naslov rada na engleskom jeziku}
\begin{abstract}
Sažetak na engleskom jeziku.

\vspace{15pt}
\noindent \textbf{Keywords:} <popis ključnih riječi na engleskom>
\end{abstract}

\end{document}
\end{verbatim}

Predložak se temelji na \emph{report} razredu te su gradivni elementi sadržaja
dokumenta (hijerarhijski navedeno): \emph{chapter}, \emph{section}, \emph{subsection} i
\emph{subsubsection}.

O stilu i strukturi dokumenta pogledajte poglavlje \ref{ch:stil-i-struktura}, a
za upoznavanje s \LaTeX--om te nekim njegovim posebnostima i receptima pogledajte
poglavlje \ref{ch:latex-upute}

Napravljeni predložak sadrži rješenje za dodavanje sažetka rada i posvete te
naredbu \texttt{engl}. Za navedeno pogledajte odjeljke \ref{sec:sazetak},
\ref{sec:posveta} i \ref{sec:engl}

Za pakete uključene u sam predložak pogledajte polja \texttt{RequirePackage} u
datoteci \texttt{završni.cls}.

\chapter{Stil i struktura}
\label{ch:stil-i-struktura}
% TODO: O stilu iz službenih MS Word uputa i općenito

\chapter{Upute za \LaTeX}
\label{ch:latex-upute}
U nastavku su dane upute za \LaTeX, no sužene za potrebe brzog uvoda u rad s
\LaTeX--om i konkretnim predloškom.

Ako se želite bolje upoznati s \LaTeX--om, konzultirajte poveznice navedene u
dodatku \ref{ch:korisne-stranice} te knjige poput \citep{oetiket2007lshort}.

\section{Instalacija \LaTeX{} prevoditelja}
\LaTeX{} dokumenti (dokumenti s \texttt{tex} ekstenzijom) se prevode krajnji
format (npr.\ \texttt{pdf}), tj.\ jezgra produkcije dokumenata je \TeX{}
prevoditelj.\footnote{\LaTeX{} je zapravo samo jedan (malo veći) dodatak
\TeX--a.}

Postoji više prevoditelja (tj.\ distribucija), no za Microsoft Windowse se
preporučuje Mik\TeX{} distribucija dostupna na \url{http://www.miktex.org/}, a za
Linux \TeX{} Live dostupan na \url{http://www.tug.org/texlive/} (koji je dostupan
i za Windowse). Većina Linux distribucija dolazi pripremljena za rad sa
\LaTeX--om, ili je prevoditelja izuzetno jednostavno instalirati preko sistemskog
upravitelja paketima (apt, yum, pacman, \ldots). Za Mik\TeX{} je bitno napomenuti
da samo najnovija verzija može skidati nove pakete, tako da ako se nađete u
prilici da vam Mik\TeX{} odbija skinuti novi paket koji želite koristiti,
najvjerojatnije je problem do verzije.


\subsection{Prevođenje \LaTeX{} dokumenta}
% TODO: Napomena kako se zbog nekih stvari prevođenje mora pokrenuti više puta
\LaTeX{} dokumenti su obične tekstualne datoteke s ekstenzijom \texttt{tex}
čiji sadržaj poštiva pravila \TeX{} sintakse.
% TODO: Dovršiti..

\section{Razvojna okruženja}
Za pisanje \LaTeX{} dokumenata dovoljan je bilo koji tekst editor ---
primjerice Notepad++ (\url{http://notepad-plus.sourceforge.net/uk/site.htm}) za
Windowse ili Vim (\url{http://www.vim.org/}) za Linux.

Postoji veliki broj \LaTeX{} razvojnih okruženja koja pokušavaju ubrzati rad
korisnika. Tu je bitno napomenuti da većina \LaTeX{} razvojnih okruženja na
Windowsima ne podržava UTF--8 kodnu stranicu!

Među razvojnim okruženjima mogu se istaknuti \emph{Texmaker} i \emph{Eclipse} sa
\emph{Texlipse} dodatkom. Navedena razvojna okruženja su besplatna, dostupna
za sve poznatije operacijske sustave te podržavaju UTF--8 kodnu stranicu.
\begin{itemize}
  \item Texmaker -- \url{http://www.xm1math.net/texmaker/}
  \item Eclipse -- \url{http://www.eclipse.org/downloads/} (bilo koji paket);
  Texlipse -- \url{http://texlipse.sourceforge.net/manual/installation.html}.
\end{itemize}

Potrebno je podesiti razvojna okruženja na ispravne putanje od alata iz \TeX{}
distribucije, te podesiti kodnu stranicu dokumenata na UTF--8.

Za Texmaker potrebno je ići \emph{Options} $\Rightarrow$ \emph{Configure
Texmaker} $\Rightarrow$ \emph{Editor} $\Rightarrow$ \emph{Editor Font Encoding}
postaviti na UTF--8. Putanje do alata \TeX{} distribucije postavljaju se u
odjeljku \emph{Commands}.

Texlipse postavke se podešavaju unutar Eclipsea. Potrebno je ići na \emph{Window}
$\Rightarrow$ \emph{Preferences} $\Rightarrow$ \emph{Texlipse} $\Rightarrow$
\emph{Builder Settings} te tu podesiti putanje od \TeX{} alata. Kodna stranica
se može podesiti za svaki projekt ili se može iskoristiti uobičajena kodna
stranica. Preporuka je podesiti uobičajenu kodnu stranicu na UTF--8. Za
podešavanje uobičajene kodne stranice potrebno je ići \emph{Window} $\Rightarrow$
\emph{Preferences} $\Rightarrow$ \emph{General} $\Rightarrow$ \emph{Workspace}
$\Rightarrow$ \emph{Text file encoding} postaviti na UTF--8. Za podešavanje
kodne stranice pojedinog projekta, potrebno je otvoriti postavke projekta
\engl{Preferences} te ići \emph{Resource} $\Rightarrow$ \emph{Text file encoding}.

\TeX{} distribucije nude više prevoditelja, \emph{latex}, \emph{pslatex} i
\emph{pdflatex}. Svaki prevoditelj prevodi u svoj format te je preporuka
koristiti direktno \emph{pdflatex} (inače je potrebno koristiti alate kao što
su \emph{ps2pdf} ili \emph{dvipdf} za dobivanje pdf dokumenta kao izlaza).


\section{Dodavanje popisa sadržaja, slika i tablica}
Sadržaj se dodaje naredbom \verb|\tableofcontents| na mjestu gdje ga želite
prikazati. Analogno tome, popis slika se dodaje naredbom \verb|\listoffigures|,
a popis tablica naredbom \verb|\listoftables|.


\section{Osnovno \LaTeX{} formatiranje}
% TODO: emph, texttt, bold, crtice, razmaci i ``~'', navodnici, trotočka,
% prelazak u novi red, paragrafi i sl., komentari, znakovi koje treba
% escapeati, gradivni elementi sadržaja (chapter, section, subsection,
% subsubsection)

% Because TeX usually assumes that periods are used to end sentences, by default it
% puts extra whitespace after each period that is followed by a space character in
% the input. To defeat this heuristic, when a non-sentence-ending period is to be
% followed by a space, the space must be an explicit blank. For example "Smith et
% al.\ claim that...".
% The converse of the previous problem happens when a capital letter precedes a
% sentence-ending period in the input. For example "...using the Pentium III. This
% means that...". In this case LaTeX assumes that the period terminates an
% abbreviation and follows it with inter-word space rather than inter-sentence
% space. Fix this by writing "...Pentium III\@. This...".


\section{Dodavanje fusnota i referenci}
Fusnote se dodaju naredbom \verb|\footnote{}|, npr.:
\begin{verbatim}
Popis fusnota\footnote{Objašnjenja koja se prikazuju
na dnu stranice} nije potreban.
\end{verbatim}
Rezultat navedenoga teksta je: ``Popis fusnota\footnote{Objašnjenja koja se
prikazuju na dnu stranice} nije potreban.''

Reference služe za povezivanje s nekim dijelom rada. Da bi ste omogućili
referenciranje na neki dio rada, tom dijelu morate postaviti oznaku
\engl{label} naredbom \verb|\label{}|, npr.:
\begin{verbatim}
\section{Dodavanje posvete}
\label{sec:posveta}
\end{verbatim}
referencirate se pomoću naredbe \verb|\ref{}|, npr.:
\begin{verbatim}
Za navedeno pogledajte odjeljke \ref{sec:sazetak},
\ref{sec:posveta} i \ref{sec:engl}
\end{verbatim}

Oznaku za referenciranje možete postaviti većini elemenata, a najčešće se
koristi za označavanje poglavlja, odjeljaka, jednadžbi, tablica i slika.


\section{Dodavanje listi, nabrajanja i opisa}
Liste tvore u okolini \texttt{itemize}, nabrajanja u okolini
\texttt{enumerate}, a opisi u okolini \texttt{description}. Primjeri:

\begin{description}
  \item[Obična lista] -- kod:
\begin{verbatim}
\begin{itemize}
  \item prva stavka,
  \item druga stavka.
\end{itemize}
\end{verbatim}
Rezultat:
\begin{itemize}
  \item prva stavka,
  \item druga stavka.
\end{itemize}

  \item[Lista s više razina] -- kod:
\begin{verbatim}
\begin{itemize}
  \item prva stavka,
  \item druga stavka,
  \begin{itemize}
    \item druga razina. 
  \end{itemize}
\end{itemize}
\end{verbatim}
Rezultat:
\begin{itemize}
  \item prva stavka,
  \item druga stavka,
  \begin{itemize}
    \item druga razina. 
  \end{itemize}
\end{itemize}

  \item[Nabrajanje] -- kod:
\begin{verbatim}
\begin{enumerate}
  \item prva stavka,
  \item druga stavka.
\end{enumerate}
\end{verbatim}
Rezultat:
\begin{enumerate}
  \item prva stavka,
  \item druga stavka.
\end{enumerate}

  \item[Nabrajanje s proizvoljnim brojačem] -- kod:
\begin{verbatim}
\begin{enumerate}[(a)]
  \item prva stavka,
  \item druga stavka.
\end{enumerate}
\end{verbatim}
Rezultat:
\begin{enumerate}[(a)]
  \item prva stavka,
  \item druga stavka.
\end{enumerate}

  \item[Opisi] -- kod:
\begin{verbatim}
\begin{description}
  \item[Esperanto:] najpoznatiji umjetni prirodni jezik.
  \item[Lojban:] sintaksno jednoznačan umjetni prirodni jezik.
  \item[Toki pona:] minimalistički umjetni prirodni jezik.
\end{description}
\end{verbatim}
Rezultat:
\begin{description}
  \item[Esperanto:] najpoznatiji umjetni prirodni jezik.
  \item[Lojban:] sintaksno jednoznačan umjetni prirodni jezik.
  \item[Toki pona:] minimalistički umjetni prirodni jezik.
\end{description}

\end{description}

Za izradu složenijih listi, nabrajanja i opisa iskoristite paket
\texttt{enumitem}. Odličan prikaz njegovih mogućnosti iznosi
\citep{collins2008enumitem}.


\section{Dodavanje tablica}
% TODO: Napisati, opširno, longtables paket također opisati, možda i booktabs?
% text wrapping


\section{Dodavanje slika}
% TODO: Napisati, navesti kako dodati više slika u redu (subfloat)
% TODO: Iskoristiti
% http://en.wikibooks.org/wiki/LaTeX/Floats,_Figures_and_Captions
% TODO: spomenuti inkscape, graphviz, ebb, mathematicu, matlab, ..


\section{Dodavanje matematičkih izraza}
Matematički izrazi se pišu u posebnoj okolini u koju se ulazi sa \verb|$|
\ldots \verb|$|, \verb|\(| \ldots \verb|\)| za linijske izraze te sa \verb|\[| \ldots
\verb|\]| ako želite da se ti izrazi nalaze u posebnoj liniji. Primjerice za
linijske izraze:
\begin{verbatim}
``Prva dama: $\sin^2 \varphi + \cos^2 \varphi = 1$.''
\end{verbatim}
Rezultat: ``Prva dama: $\sin^2 \varphi + \cos^2 \varphi = 1$.''\\
Za izraze koji se nalaze u posebnoj liniji:
\begin{verbatim}
\[ c^2 = a^2 + b^2 \]
\end{verbatim}
Rezultat:
\[ c^2 = a^2 + b^2 \]

U matematičku okolinu za izraze koji se nalaze u posebnoj liniji moguće je ući
sa \verb|$$| \ldots \verb|$$|, ali nije preporučljivo jer nije kompatibilno sa
nekim mogućnostima \LaTeX--a (no, unatoč tomu, često se koristi).

Također, postoji još okolina za pisanje matematičkih izraza među kojima je
bitno istaknuti okoline \texttt{equation}, koja donosi mogućnost označavanja
jednadžbi (oznake se automatski generiraju), i \texttt{align} koja omogućava
pisanje jednadžbi u više redova. Primjer za \texttt{equation} okolinu:
\begin{verbatim}
\begin{equation}
f(t)\ast g(t) = \int^{\infty}_{-\infty} f(\tau)g(t-\tau)d\tau.
\label{eq:conv}
\end{equation}
\end{verbatim}
Rezultat:
\begin{equation}
f(t)\ast g(t) = \int^{\infty}_{-\infty} f(\tau)g(t-\tau)d\tau.
\label{eq:conv}
\end{equation}
Primjer za \texttt{align} okolinu:
\begin{verbatim}
\begin{align}
a&=b+c,\label{eq:a}\\
d&=e+f+g,\\
h&=i+j.\label{eq:h}
\end{align}
\end{verbatim}
Rezultat:
\begin{align}
a&=b+c,\label{eq:a}\\
d&=e+f+g,\\
h&=i+j.\label{eq:h}
\end{align}

Osim uzastopnog prikaza više jednadžbi i poravnanja po znaku jednakosti, primjer
pokazuje da je moguće označiti svaku pojedinu jednadžbu. Primjerice,
\verb|``izraz za $h$ je dan jednadžbom \ref{eq:h}''|,
rezultira sa ``izraz za $h$ je dan jednadžbom \ref{eq:h}''.


Za dodatne informacije o pisanju matematičkih izraza u \LaTeX--u pogledajte
Internet stranice \url{http://www.math.uiuc.edu/~hildebr/tex/displays.html} i
\url{http://www.andy-roberts.net/misc/latex/latextutorial10.html} te pročitajte
kratke upute za korištenje AMS paketa
\url{ftp://ftp.ams.org/pub/tex/doc/amsmath/short-math-guide.pdf}. AMS paket je
uključen u predložak. \nocite{downes2002shortams}

%Citirati \citep{downes2002shortams}.


\section{Dodavanje programskog koda i sadržaja koji mora ostati neformatiran}
Za programski kod možete iskoristiti paket \texttt{listings}. Paket nije
uključen u predložak te ga morate dodati kako je opisano u odjeljku
\ref{sec:koristenje-dod-paketa} Paket je potrebno dodatno konfigurirati da
odgovara korištenom jeziku. Prije korištenja paketa dodajte naredbu, npr.~za
programski jezik Javu:
\begin{verbatim}
\lstset{language=Java, tabsize=2}
\end{verbatim}
Primjer korištenja:
\begin{lstlisting}
public class TempIdentificatorFactory {
	/** Pocetna brojcana vrijednost. */
	private static int num = 0;
	
	/**
	 * Generiranje unikatnih identifikatora
	 * privremenih varijabli.
	 * @return novi unikatni identifikator.
	 */
	public static String generateIdentificator() {
		String newIdn = new String(num + "_tmp");
		num++;
		return newIdn;
	}
}
\end{lstlisting}
Kod je stavljen unutar okoline \texttt{lstlisting}:\\
\verb|\begin{lstlisting}|\\*
\verb|Kod.|\\*
\verb|\end{lstlisting}|

Paket \texttt{listings} ne podržava UTF--8 kodnu stranicu te za korištenje
dijakritika morate iskoristiti neko drugo rješenje, npr.~paket
\texttt{listingsutf8} ili okolinu \texttt{verbatim} koja sprječava formatiranje.

Navedena okolina ima širu primjenu, tj.~služi za sav sadržaj za koji želite da
održi razmake i prelaske u novi red iz izvorne, \texttt{tex} datoteke, te da se
\TeX{} naredbe unutar tog sadržaja ne bi izvršile. Primjer korištenja:\\
\verb|\begin{verbatim}|\\*
\verb|Neki      tekst.|\\*
\verb|\end{verbatim}|\\
rezultat:
\begin{verbatim}
Neki      tekst.
\end{verbatim}

Okolina \texttt{verbatim} ima svoj linijski ekvivalent, \texttt{verb} naredbu.
Primjer korištenja:
\begin{verbatim}
Tekst bez naglaska \verb|\emph{ove riječi}|.
\end{verbatim}
Rezultat: ``Tekst bez naglaska \verb|\emph{ove riječi}|.''

Naredba \texttt{verb} funkcionira na način da ostavlja neobrađenim sve od
prve pojave odjeljitelja, koji se određuje kao prvi znak nakon same naredbe (u
primjeru to je znak ``|''), do njegove druge pojave.\footnote{Sličnu ideju s
delimiterima koristi \texttt{sed} naredba Unix ljuski.}


\section{Podjela sadržaja na više stupaca}
Za podjelu sadržaja na stupce koristite \texttt{multicol} paket koji je
uključen u predložak. Primjer korištenja:\footnote{Korišteni citati pripadaju
Donaldu E.~Knuthu, tvorcu \TeX--a}
\begin{verbatim}
\begin{multicols}{2}
The most important thing in the programming language is the
name. A language will not succeed without a good name. I have
recently invented a very good name and now I am looking for a
suitable language.

The hardest thing is to go to sleep at night, when there are
so many urgent things needing to be done. A huge gap exists
between what we know is possible with today's machines and
what we have so far been able to finish.
\end{multicols}
\end{verbatim}
Rezultat:
\begin{multicols}{2}
The most important thing in the programming language is the name. A language will
not succeed without a good name. I have recently invented a very good name and
now I am looking for a suitable language.

The hardest thing is to go to sleep at night, when there are so many urgent
things needing to be done. A huge gap exists between what we know is possible
with today's machines and what we have so far been able to finish.
\end{multicols}

Bitno je primjetiti da \texttt{multicol} okolina nije kompatibilna sa
\texttt{figure} i \texttt{table} okolinama. Za ubacivanje \texttt{figure}
okoline unutar \texttt{multicol} okoline potrebno je koristiti
\texttt{multipage} paket, primjerice:
\begin{verbatim}
\begin{minipage}{\linewidth} 
\vspace{10pt}
\centering% 
\includegraphics[width=0.8\linewidth]{sample-fig.jpg}% 
\figcaption{Slika unutar multicol okoline}% 
\label{fig:sample-fig}% 
\end{minipage}
\end{verbatim}

Za korištenje tablica iskoristite također \texttt{multipage} paket ili, ako imate
šire tablice za koje želite da se pojave u širini cijele stranice,
\texttt{table*} okolinu koja ima funkcionalnost \texttt{table} okoline ali se
uvijek proteže cijelom širinom stranice.

\section{Dodavanje literature}
\label{sec:literatura}
% TODO: neke stvari oko bibtexa, natbiba, npr. {{sadržaj koji ima i npr.
% \LaTeX}},
% ``_''
% u url-u i sl.
% Pogledati: http://en.wikibooks.org/wiki/LaTeX/Bibliography_Management


\subsection{Promjena načina citiranja}
Trenutni način citiranja je ``(autor, godina).'' Ako želite promijeniti stil
citiranja u ``[index]'' (npr.~[1]) tada u datoteci \texttt{zavrsni.cls}
promjenite parametre \emph{natbib} paketa:
\begin{verbatim}
\RequirePackage[authoryear, round]{natbib}
\end{verbatim}
u
\begin{verbatim}
\RequirePackage[numbers, square]{natbib}
\end{verbatim}

\section{Dodavanje indeksa}
% TODO: Napisati
% http://www.tex.ac.uk/cgi-bin/texfaq2html?label=makeindex

\section{Dodatci dokumenta \engl{appendix}}
Dodavanje dodatak se vrši dodavanjem poglavlja nakon naredbe \verb|\appendix|.
Ta poglavlja se označavaju velikim latiničnim slovima. Naredba \verb|\appendix|
dolazi nakon literature (vidi odjeljak \ref{sec:literatura}). Primjer:
\begin{verbatim}
\appendix
\chapter{Korisne Internet stranice o \LaTeX--u}
\end{verbatim}
Primjer rezultira poglavljem:\\*
``Dodatak A\\*
Korisne Internet stranice o \LaTeX--u''


\section{Dodavanje sažetka rada}
\label{sec:sazetak}
Sažetak rada je obavezan dio koji dolazi na sam kraj rada. Za sažetak na
Hrvatskom jeziku koristite okolinu \engl{environment} \emph{sazetak}, a za
sažetak na Engleskom jeziku okolinu \emph{abstract}. Prije korištenja
\emph{abstract} okoline potrebno je navesti naslov na engleskom jeziku naredbom
\verb|\engtitle{}|. Primjer korištenja:
\begin{verbatim}
\begin{sazetak}
Sažetak rada.

\vspace{15pt}
\noindent \textbf{Ključne riječi:} <popis ključnih riječi>
\end{sazetak}

\engtitle{Naslov rada na engleskom jeziku}
\begin{abstract}
Sažetak na engleskom jeziku.

\vspace{15pt}
\noindent \textbf{Keywords:} <popis ključnih riječi na engleskom>
\end{abstract}
\end{verbatim}
Nakon sažetka dolazi kraj dokumenta, odnosno naredba:
\begin{verbatim}
\end{document}
\end{verbatim}


\section{Dodavanje posvete}
\label{sec:posveta}
Posveta se dodaje naredbom \verb|\posveta{}| na mjestu gdje želite stranicu s
posvetom (npr.~prije tablice sadržaja --- iznad \verb|\tableofcontents|
naredbe). Primjer korištenja:
\begin{verbatim}
\posveta{Svima koji žele naučiti koristiti \LaTeX{}.}
\end{verbatim}


\section{Korištenje naredbe \texttt{engl}}
\label{sec:engl}
Naredba \verb|\engl{}| služi navođenju engleskog
prijevoda nekog termina (radi se o nestandardnoj naredbi), npr.:
\begin{verbatim}
Dodatak dokumenta \engl{appendix}.
\end{verbatim}
Rezultat je: ``Dodatak dokumenta \engl{appendix}.''


\section{Korištenje dodatnih paketa}
\label{sec:koristenje-dod-paketa}
\LaTeX{} paketi se mogu promatrati kao dodatne biblioteke u programskim
jezicima. Dodavanje novih paketa vrši se \verb|\usepackage{}| naredbom.
Navedena naredba dolazi nakon \verb|\documentclass{}| naredbe. Primjer
korištenja:
\begin{verbatim}
\documentclass{zavrsni}
\usepackage{algorithmic}
\usepackage{algorithm}
\usepackage{listings}
\usepackage{longtable}
\usepackage{multicol}
\end{verbatim}

\LaTeX{} distribucije dolaze sa velikim brojem već instaliranih paketa.
Također, distribucije je moguće podesiti da, u slučaju korištenja paketa koji
nije instaliran, same dohvate paket sa CTAN--a i instaliraju ga.

Za više informacija o \LaTeX{} paketima, konzultirajte CTAN
(\url{http://tug.ctan.org/}). Korisne upute za korištenje i instalaciju \LaTeX{}
paketa možete naći na \url{http://en.wikibooks.org/wiki/LaTeX/Packages}.

Bitno je spomenuti da se uz svaki \TeX{} paket sa CTAN--a može skinuti pripadna
dokumentacija.


\section{Provjera pravopisa}
Programi \emph{aspell} i \emph{ispell} omogućavaju jednostavnu provjeru
pravopisa \LaTeX{} dokumenata (razumiju sintaksu \TeX{}--a). Primjer korištenja:
\begin{verbatim}
aspell -c dokument.tex
\end{verbatim}
ili
\begin{verbatim}
ispell dokument.tex
\end{verbatim}
Za više informacija, instalaciju i podešavanje navedenih alata konzultirajte:
\begin{itemize}
  \item \url{http://aspell.net/}
  \item \url{http://www.gnu.org/software/ispell/ispell.html}
  \item \url{http://cvs.linux.hr/spell/}
  \item \url{http://aspell.net/win32/}
  \item \url{http://gustav.fesb.hr/hr/ispell.html}
  \item \url{http://en.wikipedia.org/wiki/GNU_Aspell}
\end{itemize}


\bibliography{literatura}
\bibliographystyle{plainnat}


\appendix
\chapter{Korisne Internet stranice o \LaTeX--u}
\label{ch:korisne-stranice}
% TODO: Ovo nekako ljepše složiti
\textbf{LaTeX – A document preparation system:}
\begin{itemize}
  \item \url{http://www.latex-project.org/}
  \item Službena stranica \LaTeX{} projekta.
\end{itemize}
\textbf{The Comprehensive TeX Archive Network (CTAN):}
\begin{itemize}
  \item \url{http://www.ctan.org/}
  \item Osnovna arhiva paketa i materijala vezanih uz \TeX~sustav.
\end{itemize}
\textbf{TeX Frequently Asked Questions on the Web:}
\begin{itemize}
  \item \url{http://www.tex.ac.uk/cgi-bin/texfaq2html}
  \item Veliki \TeX~i \LaTeX{} FAQ sa konkretnim problemima i poveznicama na
  korisne stranice.
\end{itemize}
\textbf{The Not So Short Introduction to LATEX2e:}
\begin{itemize}
  \item \url{http://tobi.oetiker.ch/lshort/lshort.pdf}
  \item Izuzetno hvaljena knjiga o \LaTeX--u prigodna za početnike.
\end{itemize}
\textbf{Šime Ungar -- Ne baš tako kratak uvod u TeX s naglaskom na LaTeX2e:}
\begin{itemize}
  \item \url{http://web.math.hr/~ungar/lkratko2e_internet.pdf}
  \item Besplatna knjiga o \LaTeX--u na Hrvatskom jeziku.
\end{itemize}
\textbf{\LaTeX{} wikibook:}
\begin{itemize}
  \item \url{http://en.wikibooks.org/wiki/LaTeX}
  \item Wiki stranica sa objašnjenjima i kvalitetnim \LaTeX{} receptima.
\end{itemize}
\textbf{\LaTeX{} courseware:}
\begin{itemize}
  \item \url{http://ahyco.ffri.hr/seminari2007/latex/home.html}
  \item Odličan \emph{courseware} pod nazivom \LaTeX{} nastao u okviru kolegija
  ``Metodika nastave informatike II'' studijske grupe Matematika i informatika Filozofskog
  fakulteta u Rijeci.
\end{itemize}
\textbf{Detexify$^2$ -- \LaTeX{} symbol classifier:}
\begin{itemize}
  \item \url{http://detexify.kirelabs.org/classify.html}
  \item Interaktivni prepoznavatelj simbola. Iznimno koristan alat ako tražite
  neki simbol.
\end{itemize}
\textbf{IEEE predavanje -- Uvod u \LaTeX:}
\begin{itemize}
  \item \url{http://www.fer.hr/ieee?@=g4ct}
  \item Predavanje koje je u travnju 2007.~godine održao mr.~sc.~Tomislav
  Petković. Popratni materijali koji se mogu naći u repozitoriju su izuzetno
  korisni.
\end{itemize}
\textbf{\TeX{}ample TikZ and PGF:}
\begin{itemize}
  \item \url{http://www.texample.net/tikz/}
  \item Primjeri korištenja \texttt{tikz} paketa za izradu izuzetno složenih
  dijagrama.
\end{itemize}
\textbf{Getting to grips with \LaTeX{}:}
\begin{itemize}
  \item \url{http://www.andy-roberts.net/misc/latex/}
  \item \LaTeX{} tutoriali s velikim brojem primjera i objašnjenjima raznih
  parametara često korištenih naredbi.
\end{itemize}
\textbf{Art of problem solving \LaTeX{} wiki:}
\begin{itemize}
  \item \url{http://www.artofproblemsolving.com/Wiki/index.php/LaTeX:About}
  \item \LaTeX{} wiki sa konkretnim primjerima.
\end{itemize}
\textbf{IMAGE Lab, University of Florida, College of Liberal Arts \& Sciences
\LaTeX{} overview:}
\begin{itemize}
  \item \url{http://www.image.ufl.edu/help/latex/}
  \item Primjeri i upute za korištenje \LaTeX{}--a temeljeni na knjigama \emph{A
  Beginner's Introduction to Typesetting with LaTeX} i \emph{The Not So Short
  Introduction to LaTeX2e}.
\end{itemize}
\textbf{The Comprehensive \LaTeX{} Symbol List}
\begin{itemize}
  \item \url{http://www.ctan.org/tex-archive/info/symbols/comprehensive/symbols-letter.pdf}
  \item Lista 5913 simbola dostupna iz raznih paketa na CTAN--u.
\end{itemize}
\textbf{The Visual \LaTeX{} FAQ}
\begin{itemize}
  \item \url{http://mirror.ctan.org/info/visualFAQ/visualFAQ.pdf}
  \item Izuzetno dobro napravljen skup \LaTeX{} recepata koji se temelji na
  konkretnom primjeru recepta i poveznicom na Internet stranicu sa detaljnim
  opisom recepta.
\end{itemize}


% TODO: Napisati sažetak.
\begin{sazetak}
Sažetak rada.

\vspace{15pt}
\noindent \textbf{Ključne riječi:} Završni rad, Seminar, \LaTeX
\end{sazetak}

\engtitle{Manual for bachelor thesis and Seminar \LaTeX{} templates}
\begin{abstract}
Sažetak na engleskom jeziku.

\vspace{15pt}
\noindent \textbf{Keywords:} bachelor thesis, seminar, \LaTeX
\end{abstract}

\end{document}
