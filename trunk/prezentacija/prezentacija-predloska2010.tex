\documentclass{beamer}
\usepackage[utf8]{inputenc}
\usepackage[croatian]{babel}
\usepackage[T1]{fontenc}
\usepackage{multicol}
\usepackage{listings}
\usepackage{booktabs}

\usetheme[pageofpages=od,% String used between the current page and the
                         % total page count.
          bullet=circle,% Use circles instead of squares for bullets.
          titleline=true,% Show a line below the frame title.
          alternativetitlepage=true,% Use the fancy title page.
          titlepagelogo=FER_logo.jpg,% Logo for the first page.
          ]{Torino}

\title{\LaTeX\ -- prvi koraci}
\author{Ivan Krišto}
\institute{Fakultet elektrotehnike i računarstva, Zagreb}
\date{22.~travnja, 2010.}

\begin{document}

\begin{frame}[t,plain]
\titlepage
\end{frame}

\begin{frame}[t]{Organizacija}
\textbf{Cilj:} dati kratke upute i savjete oko izrade dokumenata u \LaTeX-u te
objasniti način korištenja predložaka za pismene radove.

\pause
\begin{multicols}{2}
\begin{itemize}
  \item Što je \LaTeX{}?
  \item Kako \LaTeX{} radi
  \item Prednost nad tekst procesorima
  \item Razvojna okruženja
\pause
  \item Osnovno o \LaTeX-u:
  \begin{itemize}
    \item Struktura \LaTeX{} dokumenta
    \item Naredbe i okoline
    \item Liste, nabrajanja i opisi
    \item Tablice
    \item Dodavanje slika
    \item Pisanje matematičkih izraza
    \item Uključivanje neformatiranog sadržaja 
    \item Literatura
    \item Paketi
    \item Specifičnosti, problemi i savjeti
  \end{itemize}
\pause
  \item Kako koristiti predložak
\end{itemize}
\end{multicols}
\end{frame}

\begin{frame}[t]{Što je \LaTeX{}?}
\begin{itemize}
  \item Sustav za pripremu dokumenata i \emph{slovoslagarstvo} (engl.\ \emph{typesetting})
  \item Nadogradnja \TeX-a
  \item Usmjeravanje posla na strukturu i sadržaj (ne izgled)
  \item Jednostavna izrada lijepih dokumenata po predefiniranim stilovima
  \item Izuzetno velike mogućnosti
  \item Veći stilski zahvati znaju biti izuzetno teški
  \item Nije zamišljen kao WYSIWYG alat
\end{itemize}
\end{frame}

\begin{frame}[t]{Kako \LaTeX{} radi}
\begin{figure}[htb]
\begin{center}
\begin{picture}(270,15)
\put(3,5){\makebox(40,15){\texttt{dokument.tex}}}
\put(58,12){\vector(1,0){15}}
\put(75,5){\framebox(90,15){\texttt{pdflatex}}}
\put(167,12){\vector(1,0){15}}
\put(197,5){\makebox(40,15){\texttt{dokument.pdf}}}
\end{picture}
\caption{Tok \LaTeX\ prevođenja}
\label{fig:ru-sustav-dijagram}
\end{center}
\end{figure}
\begin{itemize}
  \item \texttt{dokument.tex} -- tekstualni dokument sa sadržajem i \LaTeX\ naredbama
  \item \texttt{pdflatex} -- prevođenje \texttt{tex} dokumenta u \texttt{pdf}\\
  (prevoditelj \texttt{latex} prevodi u \texttt{dvi})
  \pause
  \item Distribucije:
  \begin{itemize}
    \item Windowsi -- Mik\TeX{} (\url{http://miktex.org/})
    \item Linux -- \TeX\ Live (\url{http://www.tug.org/texlive/}; dostupan također za Windowse i MacOS)
    \item Sadrže više prevoditelja i skup korisnih alata (\texttt{ps2pdf}, \texttt{ebb}, \ldots)
  \end{itemize}
\end{itemize}
\end{frame}

\begin{frame}[t]{Prednost nad tekst procesorima}
\begin{itemize}
  \item Odjeljivanje sadržaja i strukture od stila
  \item Dokumenti se pišu kao obične tekstualne datoteke s \TeX\ sintaksom
  \item Mogućnost korištenja sustava za verziranje (VCS sustavi -- SVN, Mercurial, Git, \ldots)
  \item Mogućnost pisanja dokumenata u dijelovima (naredba \texttt{include})
  \item Dokumenti su prenosivi kroz verzije \LaTeX-a (primjerice, dokument napisan MS Wordom 2007 nije čitljiv u MS Wordu 2000)
  \item Trudi se poštivati osnove tipografije
\end{itemize}
\end{frame}

\begin{frame}[t]{Razvojna okruženja}
\begin{itemize}
\item TexMaker (\url{http://www.xm1math.net/texmaker/})
\begin{description}
  \item[+] Veliki broj gotovih primjera (odlično za početnike!)
  \item[--] Pretrpano sučelje
\end{description}
\pause
\item \TeX{}works (\url{http://code.google.com/p/texworks/})
\begin{description}
  \item[+] Jednostavan i malen
  \item[+] Ugrađeni preglednik
  \item[--] Nema gotovih primjera (simbola i konstrukcija)
\end{description}
\pause
\item Eclipse \& \TeX{}lipse (\url{http://www.eclipse.org/} \& \url{http://texlipse.sourceforge.net/})
\begin{description}
  \item[+] Najmoćnije razvojno okruženje
  \item[+] Odličan prikaz grešaka (u razini linije!)
  \item[+] Autocomplete čita čak i \texttt{bib} datoteke
  \item[+] Prevodi paralelno s radom
  \item[--] Glomazan
\end{description} 
\end{itemize}
\end{frame}

\begin{frame}[fragile]
\frametitle{Struktura \LaTeX{} dokumenta}
\begin{multicols}{2}
Osnovni \LaTeX\ dokument:
\begin{verbatim}
\documentclass{report}
\title{Naslov}
\author{Autor}
\begin{document}
\chapter{Uvod}
Uvod..
\section{Unutar uvoda}
Prvi paragraf.

Drugi paragraf.
\end{document}
\end{verbatim}
\pause
\begin{itemize}
  \item Stil se definira \verb|\documentclass| naredbom
  \item Sadržaj se nalazi u \texttt{document} okolini
  \item Struktura se gradi \verb|\chapter|, \verb|\section|, \verb|\subsection| i
  \verb|\subsubsection| naredbama te podjelom na paragrafe 
  \item Tekst se odjeljuje u paragrafe praznim redom
  \item Prelazak u novi red (bez stvaranja paragrafa) -- sekvenca: \verb|\\|
\end{itemize}
\end{multicols}
\end{frame}

\begin{frame}[fragile]
\frametitle{Naredbe i okoline}
\begin{description}
  \item[Naredbe:] Elementi \LaTeX-a koji primaju argumente i parametre te se transformiraju u
  oblikovani tekst
  \begin{itemize}
    \item Sintaksa: \verb|\naziv[parametar][...]{argument}{...}|
    \item Primjeri: \verb|\tableofcontents|, \verb|\emph{riječ}|, \verb|\frac{2}{x+2}|, \verb|\usepackage[croatian]{babel}|
  \end{itemize}
  \item[Okoline:] Slično kao naredbe, ali umjesto argumenta primaju
  segment dokumenta (razlika je u obradi paragrafa)
  \begin{itemize}
    \item Sintaksa: \verb|\begin{naziv}[parametar][..]{argument}{..}| \ldots \verb|\end{naziv}|
    \item Primjeri: \verb|\begin{document}| \ldots \verb|\end{document}|, \verb|\begin{tabular}{ l c r }| \ldots \verb|\end{tabular}|
  \end{itemize}
\end{description}
\end{frame}

\begin{frame}[fragile]
\frametitle{Liste, nabrajanja i opisi}
\begin{multicols}{2}
\begin{itemize}
  \item Liste:
  \begin{verbatim}
  \begin{itemize}
    \item kruh
    \item mlijeko
    \item jaja
  \end{itemize}
  \end{verbatim}
  \item Nabrajanja:
  \begin{verbatim}
  \begin{enumerate}
    \item Prva
    \item Druga
    \item Treća
  \end{enumerate}
  \end{verbatim}
  \item Opisi:
  \begin{verbatim}
  \begin{description}
    \item[Pingvin] ptica
    \item[Špar] riba
  \end{description}
  \end{verbatim}
  \item Gniježđenje:
  \begin{verbatim}
  \begin{itemize}
    \item Prva stavka
    \begin{enumerate}
      \item Numerirano
    \end{enumerate}
  \end{itemize}
  \end{verbatim}
\end{itemize}
\end{multicols}
\end{frame}

\begin{frame}[fragile]
\frametitle{Dodavanje slika}
\begin{multicols}{2}
\begin{itemize}
  \item Dodaju se pomoću \texttt{figure} lebdeće (engl.\ \emph{floating}) okoline
  (slika se ne mora pojaviti tamo gdje je navedete!)
\begin{verbatim}
\begin{figure}[htb]
\centering
\includegraphics[width=2cm]
                {FER_logo.jpg}
\caption{Logo FER-a}
\label{fig:fer-logo}
\end{figure}
\end{verbatim}
\end{itemize}
\begin{figure}[htb]
\centering
\includegraphics[width=2cm]{FER_logo.jpg}
\caption{Logo FER-a}
\label{fig:fer-logo}
\end{figure}
\end{multicols}
\begin{itemize}
  \item \texttt{htb} -- pozicioniranje (\emph{here}, \emph{top}, \emph{bottom});
  \texttt{caption} -- naslov (uvijek na dnu); \texttt{label} -- služi za
  referenciranje pomoću \texttt{ref} naredbe
\end{itemize}
\end{frame}

\begin{frame}[fragile]
\frametitle{Tablice (1)}
\begin{itemize}
  \item Ostvaruju se preko \texttt{table} i \texttt{tabular} okolina
  \item \texttt{table} je lebdeća okolina unutar koje se
  smješta \texttt{tabular} okolina s podatcima
  \item Znak \& odvaja stupce; sekvenca \textbackslash\textbackslash\ odvaja retke 
\begin{verbatim}
\begin{table}[htb]
\caption{Konstante}
\centering
\begin{tabular}{llr} \hline
Konstanta & Opis & Vrijednost \\ \hline
$\pi$ & Pi & 3.14159 \\
$e$ & Eulerov broj & 2.71828 \\
$\varphi$ & Zlatni rez & 1.61803 \\ \hline
\end{tabular}
\end{table}
\end{verbatim}
\end{itemize}
\end{frame}

\begin{frame}[fragile]
\frametitle{Tablice (2)}
\begin{table}[htb]
\caption{Konstante}
\centering
\begin{tabular}{llr} \hline
Konstanta & Opis & Vrijednost\\ \hline
$\pi$ & Pi & 3.14159 \\
$e$ & Eulerov broj & 2.71828 \\
$\varphi$ & Zlatni rez & 1.61803 \\ \hline
\end{tabular}
\end{table}
\begin{table}[htb]
\caption{Konstante (\texttt{booktabs} paket)}
\centering
\begin{tabular}{llr} \toprule
Konstanta & Opis & Vrijednost\\ \midrule
$\pi$ & Pi & 3.14159 \\
$e$ & Eulerov broj & 2.71828 \\
$\varphi$ & Zlatni rez & 1.61803 \\ \bottomrule
\end{tabular}
\end{table}
\end{frame}

\begin{frame}[fragile]
\frametitle{Pisanje matematičkih izraza}
\begin{itemize}
  \item Brzo i jednostavno (uz malo vježbanja)
  \item Piše se u modu za matematiku
  \begin{itemize}
    \item \verb|$ ... $| za izraze u istoj liniji uz tekst
    \item \verb|\[  ... \]| (ili \verb|$$  ... $$|) za izraze u posebnoj liniji
  \end{itemize}
  \item Primjeri: \verb|$x^2$| -- $x^2$, \verb|$\int_{2}^{10} x^2 dx$| -- $\int_{2}^{10} x^2 dx$
  \[  f(t)\ast g(t) = \int^{\infty}_{-\infty} f(\tau)g(t-\tau)d\tau \]
\begin{verbatim}
\[   f(t)\ast g(t) = \int^{\infty}_{-\infty}
                     f(\tau)g(t-\tau)d\tau   \]
\end{verbatim}
  \item Dobre upute za pisanje matematičkih izraza u \LaTeX-u:
  \url{ftp.ams.org/pub/tex/doc/amsmath/short-math-guide.pdf}
\end{itemize}
\end{frame}

\lstset{language=C, tabsize=2}
\begin{frame}[fragile]
\frametitle{Uključivanje neformatiranog sadržaja}
\begin{multicols}{2}
\begin{itemize}
  \item Neformatirani sadržaj -- tekst koji se mora prikazati s očuvanim izvornim razmacima, redovima i neekspandiranim naredbama
  \item Linijski: naredba \verb|\verb|, primjer: ``\verb|Primjer    \verb naredbe|''
  \item Višelinijski: okolina \texttt{verbatim}:
  \begin{verbatim}
  Linija prva,
    druga
      i treća.
  \end{verbatim}
  \item Programski kod se može uključiti unutar \texttt{verbatim} okoline
  \item Posebni paketi za uključivanje programskog koda, primjerice \texttt{listings}
\end{itemize}
  \begin{lstlisting}
  int fib(int n) {
   if (n <= 0) return 0;
   else if(n == 1)return 1;
   else return fib(n-1)
  	          + fib(n-2);
  }
  \end{lstlisting}
\end{multicols}
\end{frame}

\begin{frame}[fragile]
\frametitle{Literatura}
\begin{itemize}
  \item Upravljanjem literature bavi se Bib\TeX
  \item Literatura se navodi u \texttt{bib} datoteci (Bib\TeX\ zapisi)
\begin{verbatim}
@article{greenwade93,
 author = "George D. Greenwade",
 title = "The {C}omprehensive {T}ex {A}rchive {N}etwork",
 year = "1993",
 journal = "TUGBoat"
}
\end{verbatim}
\item Veza \texttt{literatura.bib} i \texttt{tex} datoteka: \verb|\bibliography{literatura}|
\item Postavljanje stila popisa literature \verb|\bibliographystyle{plain}|% (stil ``\texttt{plain}'')
\item Referenciranje na literaturu, primjerice: \verb|\cite{greenwade93}|
\item Prevođenje: \texttt{pdflatex}, \texttt{bibtex}, \texttt{pdflatex}, \texttt{pdflatex}
\item \texttt{natbib} paket -- sortiranje, stil, referenciranje preko \verb|\citep|, \verb|\citet| 
\end{itemize}
\end{frame}

\begin{frame}[t]{Paketi}
\begin{itemize}
  \item Skupovi definicija naredbi i okolina
  \item Glavni izvor: ``The Comprehensive \TeX\ Archive Network (CTAN)''
  \item Ručna instalacija je \emph{relativno} složena
  \item \LaTeX\ distribucije uglavnom same skinu i instaliraju paket koji zatražite
  \item Dodaju se \texttt{usepackage} naredbom odmah nakon \texttt{documentclass} naredbe
  \item Par korisnih paketa:
  \begin{description}
	\item[natbib] olakšavanje rada s literaturom
	\item[booktabs] naredbe za stiliziranje tablica
	\item[enumitem] proširenje mogućnosti listi i nabrajanja
	\item[amsmath] proširenje mogućnosti za pisanje matematičkih izraza
	\item[multicol] naredbe za pisanje u stupcima
	\item[hyperref] omogućava interaktivne poveznice (reference)
  \end{description}
\end{itemize}
\end{frame}

\begin{frame}[fragile]
\frametitle{Specifičnosti, problemi i savjeti}
\begin{itemize}
  \item Posebni znakovi (osnova \TeX\ sintakse): \$ \% \_ \} \{ \& \# \textbackslash
  \begin{itemize}
    \item Koristi se isključivanje (engl.\ \emph{escape}), primjerice \verb|\$|, a \textbackslash\ se dobiva naredbom \verb|\textbackslash|
  \end{itemize}
  \item Nakon točke \LaTeX\ doda malo više razmaka -- pretpostavlja se kraj rečenice; ``\verb|\ |'' nakon točke stvara običan razmak:
\begin{verbatim}
prof. dr. sc. Donald E. Knuth\\
prof.\ dr.\ sc.\ Donald E.\ Knuth
\end{verbatim}
prof. dr. sc. Donald E. Knuth\\
prof.\ dr.\ sc.\ Donald E.\ Knuth
  \item Može se dogoditi da je riječ loše prelomljena te prelazi rub stranice
  \begin{itemize}
    \item Iskoristiti \verb|\hyphenation|, primjerice: \verb|\hyphenation{pre-lo-mlje-na ra-ču-na-lo}|
  \end{itemize}
  \item Pravopis možete provjeriti pomoću \emph{aspell} ili \emph{ispell} alata, primjerice: \texttt{aspell -c dokument.tex}
\end{itemize}
\end{frame}

\begin{frame}[fragile]
\frametitle{Kako koristiti predložak (1)}
\begin{itemize}
  \item Osnova predloška je \texttt{fer.cls} datoteka (\texttt{fer.bst} -- literatura)
  \item Predložak se uključuje naredbom \verb|\documentclass{fer}|
  \item Opcije predloška (standardne postavke su označene sa ``($\ast$)''):
  \begin{itemize}
    \item \texttt{seminar} -- podesi stil za Seminar ($\ast$)
    \item \texttt{diplomski} -- podesi stil za Diplomski rad
    \item \texttt{zavrsni} -- podesi stil za Završni rad
    \item \texttt{lmodern} -- koristi font \emph{lmodern} ($\ast$)
    \item \texttt{times} -- koristi font \emph{times}
    \item \texttt{utf8} -- učitaj \texttt{inputenc} paket i opcijom \texttt{utf8} ($\ast$)
    \item \texttt{cp1250} -- učitaj \texttt{inputenc} paket i opcijom \texttt{cp1250}
    \item \texttt{authoryear} -- stil citiranja ``(author, year)'' ($\ast$)
    \item \texttt{numeric} -- stil citiranja ``[broj]''
  \end{itemize}
  \item Primjer: \verb|\documentclass[zavrsni, numeric]{fer}|
  \item Redoslijed zadavanja opcija nije bitan
\end{itemize}
\end{frame}

\begin{frame}[fragile]
\frametitle{Kako koristiti predložak (2) -- primjer: Završni rad}
\begin{multicols}{2}
\begin{verbatim}
\documentclass[zavrsni]{fer}
\begin{document}
\thesisnumber{444}
\title{Kako trenirati
       vlastitog zmaja}
\author{Ivan Krišto}
\maketitle
\izvornik
\zahvala{}
\tableofcontents

\chapter{Uvod}
Uvod rada.

\bibliography{literatura}
\bibliographystyle{fer}
\begin{sazetak}
Sažetak na hrvatskom jeziku.
\kljucnerijeci{Ključne riječi}
\end{sazetak}

\engtitle{How To Train
          Your Dragon}
\begin{abstract}
Abstract.
\keywords{Keywords.}
\end{abstract}
\end{document}
\end{verbatim}
\end{multicols}
\end{frame}

\begin{frame}[fragile]
\frametitle{Kako koristiti predložak (3) -- primjer: Seminar}
\begin{multicols}{2}
\begin{verbatim}
\documentclass[times]{fer}

\begin{document}
\title{Kako trenirati
       vlastitog zmaja}
\voditelj{Hiccup Horrendous
          Haddock}
\author{Ivan Krišto}
\maketitle

\tableofcontents

\chapter{Uvod}
Uvod rada.

\chapter{Zaključak}
Zaključak.

\bibliography{literatura}
\bibliographystyle{fer}

\chapter{Sažetak}
Sažetak.
\end{document}
\end{verbatim}
\end{multicols}
\end{frame}

\end{document}
