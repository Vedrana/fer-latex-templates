\documentclass{beamer}
\usepackage[utf8]{inputenc}
\usepackage[croatian]{babel}
\usepackage[T1]{fontenc}

\usetheme[pageofpages=od,% String used between the current page and the
                         % total page count.
          bullet=circle,% Use circles instead of squares for bullets.
          titleline=true,% Show a line below the frame title.
          alternativetitlepage=true,% Use the fancy title page.
          titlepagelogo=ktlab_logo_m_q,% Logo for the first page.
          ]{Torino}

\title{Projekt iz programske potpore 2009 -- savjeti}
\author{Ivan Krišto}
\institute{Fakultet elektrotehnike i računarstva, Zagreb}
\date{29.~rujna, 2009.}

\begin{document}

\begin{frame}[t,plain]
\titlepage
\end{frame}

\begin{frame}[t]{Projekt i tim}
\begin{itemize}
  \item Rad u timu
  \begin{itemize}
    \item Kako organizirati rad u timu, s kojim problemima se možete susresti i
    koje alate možete koristiti.
  \end{itemize} 
  \item Programiranje u timu
  \begin{itemize}
    \item Koji su problemi pri podjeli posla, malo o osnovama zdravog
    programskog oblikovanja, o problemima koje treba izbjeći te alatima koje
    možete koristiti.
  \end{itemize}
  \item Konkretni savjeti
  \begin{itemize}
    \item Par konkretnih savjeta uglavnom vezanih uz programiranje.
  \end{itemize}
\end{itemize}
\end{frame}

\begin{frame}[t]{Rad u timu (1)}
\begin{itemize}
  \item Potražite srodna rješenja!
  \begin{itemize}
    \item http://citeseerx.ist.psu.edu/
    \item http://scholar.google.com/
    \item http://sourceforge.net/
    \item \ldots
  \end{itemize}
  \item Komunikacija -- mailing lista i sastanci
  \begin{itemize}
    \item Sastanci ($D258$ ili $D269$ $\rightarrow$ ploča!)
    \item Dogovaranje termina: http://whenisgood.net/
    \item Sastati se kad god postoji nesporazum ili kasnite s poslom
    \item Izbjegavati ICQ, MSN i forume
    \item Mailing lista -- http://groups.google.com/
  \end{itemize}
\end{itemize}
\end{frame}

\begin{frame}[t]{Rad u timu (2)}
\begin{itemize}
  \item Dogovori moraju biti \emph{jasni} i \emph{zapisani}!
  \begin{itemize}
    \item Sve dogovoreno zapišite na servise poput google grupe ili google docsa
    \item Ako možete, vizualizirajte (ganttogram i sl.~dijagrami)
    \item Primjer grafičkog prikaza podjele poslova:
    http://docs.google.com/View?id=ddfk9wtr\_14fmtsbsck
  \end{itemize}
  \item Kad mailing lista nije dovoljna, a ne možete se sastati:
  \begin{itemize}
    \item Skype (voice chat)
    \item Google docs -- collaboration
    \item On-line collaborative whiteboards,
    npr.~http://www.imaginationcubed.com/
  \end{itemize}
  \item Dodatni rokovi su korisna stvar (namijenjeno asistentima i savjetnicima)
  \item Dokumentacija
  \begin{itemize}
    \item wiki (kontinuirano dokumentiranje?), \LaTeX, Google docs
  \end{itemize}
\end{itemize}
\end{frame}

\begin{frame}[t]{Rad u timu -- mogući problemi}
\begin{itemize}
  \item Član tima ne čita e-mailove ili se ne pojavljuje na sastancima
  \item Član tima ne obavlja poslove
  \item Posao je obavljen, ali loše
  \item Dokumentacija je gotova---nalazi se u $5$ word dokumenata sa različitim
  formatiranjem!
  \item Član nešto nije shvatio, a svejedno se srami pitati
  \item Rad se odgađa za vrijeme koje se očekuje da će biti slobodno
  (koncentracija rada pred rokove)
  \item Teorijska podloga nije jasna i nastavljate gubiti vrijeme proučavajući
  literaturu (pitajte asistente!)
  \item Voditelj tima prezume previše posla
  \item Voditelj tima nije fleksibilan
\end{itemize}
\end{frame}

\begin{frame}[t]{Programiranje u timu (1)}
\begin{itemize}
  \item Napravite dijagrame (npr.~UML ili kvazi-UML)
  \item Potražite gotove komponente!
  \begin{itemize}
    \item http://www.apache.org/
    \item http://www.java.net/
    \item http://www.cpan.org/
    \item http://pythonsource.com/
    \item \ldots
  \end{itemize}
  \item Podrška za grupni rad na istom kodu -- \emph{SVN} i sl.~sustavi
  \item Kontrola koda
  \begin{itemize}
    \item Issue trackeri (http://morgoth.zemris.fer.hr/trac/jcms/report/1)
    \item Code review (http://code.google.com/p/support/source/detail?r=70)
    \item Negativna strana: previše posla za relativno malo vremena (prepustiti
    savjetnicima?)
  \end{itemize} 
\end{itemize}
\end{frame}

\begin{frame}[t]{Programiranje u timu (2)}
\begin{itemize}
  \item Podijelite posao na odvojene dijelove koji komuniciraju poštujući
  programska sučelja
  \item Pri podjeli poštujte principe dobrog programskog oblikovanja (predmet
  OOUP)
  \begin{itemize}
  	\item http://www.zemris.fer.hr/\textasciitilde ssegvic/pubs/ooup0Intro.pdf
  	\item http://www.zemris.fer.hr/\textasciitilde ssegvic/pubs/ooup1Principles.pdf\\
  	(isplati se sve pročitati, no za početak krenite od slidea $63$.)
  \end{itemize}
  \item Jasno definirajte i dobro komentirajte programska sučelja!
  \begin{itemize}
    \item ``Dobro komentiranje'' -- navedena je odgovornost\footnote{Primjer
    problema -- što znači ``odgovornost''} metode, opis ulaznih argumenata,
    opis rezultata, ako je potrebno onda i primjeri
    \item Nemojte ići u detalje, ionako će se mijenjati
  \end{itemize}
  \item ``Dobro'' komentiranje nije rezervirano samo za sučelja -- koristite
  JavaDoc/nDoc/PyDoc/Haddock (ovisno o jeziku)
\end{itemize}
\end{frame}

\begin{frame}[t]{Programiranje u timu -- mogući problemi}
\begin{itemize}
  \item Netko nije poštivao propisano programsko sučelje, npr.~proširio ga (nije
  ga razumio?)
  \item GUI komunicira sa svim slojevima koda (Model--View--Controller!!!)
  \item Kod nije komentiran (a trebao je biti)---ostali imaju problema
  \item Napisane metode su presložene za korištenje čak i za ostale članove tima
  \item Diranje tuđeg koda bez prethodnog dogovora ili obavijesti
  \item Promašena ideja implementacije (osoba je trebala napraviti jednu stvar,
  no krivo je shvatila problematiku ili čekala druge)
  \item Projekt nije dobro programski dizajniran
  \item Članovi tima koriste različite kodne stranice (encoding) datoteka sa
  izvornim kodom
\end{itemize}
\end{frame}

\begin{frame}[t]{Konkretni savjeti}
\begin{itemize}
  \item GUI radite od samog početka
  \item Uz svaki SVN commit dodajte opis što ste napravili
  \item Nemojte gubiti previše vremena komentiranjem, komentira se samo ono što
  nije očito
  \item Testirajte stvari koje možete -- poslužite se raznim skriptama
  (npr.~perl/python) ili okruženjima za testiranje (npr.~jUnit)
  \item Ako se kod ne može kompajlirati, ne stavljajte ga u SVN repozitorij!\\
  (osim ako se posebno ne dogovorite)
  \item Izbjegavajte copy-paste!!!
  \item Isključivo koristite UTF-8 kodnu stranicu (encoding) za datoteke s
  izvornim kodom
\end{itemize}
\end{frame}

\begin{frame}[t]{Pitanja?}
Prijedlozi:
\begin{itemize}
  \item Neki pojmovi nisu bili jasni
  \item Neke točke treba dodatno pojasniti
  \item Koje alate koristiti za neke probleme (npr.~crtanje dijagrama)
  \item \ldots
\end{itemize}
\end{frame}

\end{document}
